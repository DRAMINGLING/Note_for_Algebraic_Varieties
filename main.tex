\documentclass{ctexart}
\usepackage[a4paper,left=3cm,right=3cm,top=3cm,bottom=3cm]{geometry}
\usepackage{amssymb,amsfonts,amsmath,amsthm}
\usepackage{anyfontsize}
\usepackage{graphicx}
\usepackage{tikz-cd}
\usepackage{bm}
\usepackage{bbm}
\usepackage{mathrsfs}
\usepackage[colorlinks = true,      % 启用彩色链接而非带框链接
            linkcolor = blue,       % 内部链接(如目录、交叉引用)颜色
            urlcolor = blue,        % 外部链接颜色
            citecolor = blue]{hyperref} % 参考文献链接颜色

\newtheorem{definition}{Definition}[section]
\newtheorem{remark}{Remark}[section]
\newtheorem{theorem}{Theorem}[section]
\newtheorem{proposition}{Proposition}[section]
\newtheorem{setting}{Setting}[section]
\newtheorem{lemma}{Lemma}[section]
\newtheorem{reason}{Reason}[section]
\newtheorem{example}{Example}[section]
\newtheorem{corollary}{Corollary}[section]
\newtheorem{exercise}{Exercise}[section]

\renewcommand{\proofname}{\bf Proof}
\renewcommand{\contentsname}{Contents}
\renewcommand{\hom}{\operatorname{Hom}}

\newcommand{\mapsfrom}{\mathrel{\reflectbox{\ensuremath{\mapsto}}}}
\newcommand{\longmapsfrom}{\mathrel{\reflectbox{\ensuremath{\longmapsto}}}}
\newcommand{\groupaction}{\mathrel{\text{\scalebox{1.5}{\rotatebox[origin=c]{90}{$\circlearrowright$}}}}}

\newcommand{\rank}{\operatorname{rank}}
\newcommand{\im}{\operatorname{im}}
\newcommand{\identity}{\operatorname{id}}
\newcommand{\trdeg}{\operatorname{trdeg}}
\newcommand{\codim}{\operatorname{codim}}
\newcommand{\depth}{\operatorname{depth}}
\newcommand{\height}{\operatorname{height}}
\newcommand{\projdim}{\dim_{\operatorname{proj}}}
\newcommand{\characteristic}{\operatorname{Char}}

\newcommand{\derivation}{\operatorname{Der}}
\newcommand{\ext}{\operatorname{Ext}}
\newcommand{\mor}{\operatorname{Mor}}
\newcommand{\aut}{\operatorname{Aut}}
\newcommand{\homend}{\operatorname{End}}

\newcommand{\sheafhom}{\mathcal{H}\kern -.5pt om}
\newcommand{\sheaftor}{\mathcal{T}\kern -.5pt or}
\newcommand{\sheafext}{\mathcal{E}\kern -.5pt xt}
\newcommand{\sheafend}{\mathcal{E}\kern -.5pt nd}

\newcommand{\coker}{\operatorname{coker}}
\newcommand{\coim}{\operatorname{coim}}
\newcommand{\cone}{\operatorname{cone}}
\newcommand{\totalcomplex}{\operatorname{Tot}}

\newcommand{\spec}{\operatorname{Spec}}
\newcommand{\proj}{\operatorname{Proj}}
\newcommand{\spf}{\operatorname{Spf}}
\newcommand{\fraction}{\operatorname{Frac}}
\newcommand{\singular}{\operatorname{Sing}}

\newcommand{\reduction}{\operatorname{red}}
\newcommand{\normalization}{\operatorname{nor}}
\newcommand{\analytification}{\operatorname{an}}

\newcommand{\inverselimit}{\underset{\longleftarrow}{\lim}} 
\newcommand{\colimit}{\underset{\longrightarrow}{\lim}}
\newcommand{\derivedotimes}{\overset{\llcorner}{\otimes}}

\newcommand{\grassmannianscheme}[2]{\operatorname{Gr}(#1, #2)}
\newcommand{\grassmanmianfunctor}[2]{\mathcal{G}\kern -.5pt r(#1, #2)}
\newcommand{\hilbertscheme}{\operatorname{Hilb}}
\newcommand{\hilbertfunctor}{\mathcal{H}\kern -.5pt ilb}
\newcommand{\picardscheme}{\operatorname{Pic}}
\newcommand{\picardfunctor}{\mathcal{P}\kern -.5pt ic}
\newcommand{\isomorphismscheme}{\operatorname{Isom}}
\newcommand{\isomorphismfunctor}{\mathcal{I}\kern -.5pt som}

\newcommand{\abeliangroups}{\mathbf{Ab}}
\newcommand{\rings}{\mathbf{Ring}}
\newcommand{\commutativerings}{\mathbf{CRing}}

\newcommand{\module}{\operatorname{Mod}}
\newcommand{\algebra}{\operatorname{Alg}}

\title{代数簇}
\author{}
\date{}

\begin{document}
\maketitle
\newpage

\tableofcontents
\newpage

\section{Affine Variety}

\subsection{Zariski topology}
\label{subsection:Affine_Variety_Zariski_topology}

\begin{definition}
    Let $k$ be a field.
    Define the affine space over $k$ of dimension $n$ to be $\mathbb{A}_{k}^{n} := k^{\oplus n}$.
\end{definition}
\begin{definition}
    Let $\mathbb{A}_{k}^{n}$ be affine space.
    Assume $S \subset k[x_{1}, \cdots, x_{n}]$ be a set of polynomials.
    Define $V(S) := \{x\in \mathbb{A}_{k}^{n} \big{|} f(x) = 0, \forall f\in S\}$.
    Such a set is called an algebraic set.
\end{definition}
\begin{lemma}
    Let $\mathbb{A}_{k}^{n}$ be affine space.
    Then algebraic sets have following properties 
    \begin{itemize}
        \item For two subset $S_{1}, S_{2} \subset k[x_{1}, \cdots, x_{n}]$, $V(S_{1}) \cup V(S_{2}) = V(S_{1}S_{2})$,
        where $S_{1}S_{2} = \{fg \big{|} f\in S_{1}, g\in S_{2}\}$
        \item For any collection $\{S_{i}\}$, $\cap_{i} V(S_{i}) = V(\cup_{i} S_{i})$
    \end{itemize}
    \label{Lemma 1.1}
\end{lemma}
\vspace{-\baselineskip} % 减少一行间距
\begin{remark}
    With this lemma, we immediately have that algebraic sets give a topological structure on $\mathbb{A}_{k}^{n}$ as closed subsets, called the Zariski topology.
\end{remark}

\subsection{Affine variety}
\label{subsection:Affine_Variety_Affine_variety}

From now on, we would rename algebraic variety in affine space as affine variety.
\begin{definition}
    Let $\mathbb{A}_{k}^{n}$ be affine space, $V \subseteq \mathbb{A}_{k}^{n}$ an affine subvariety.
    Define the associated ideal of $V$ to be $I(V) := \{f\in k[x_{1}, \cdots, x_{n}] \big{|} f(x) = 0, \forall x\in V\}$
\end{definition}
\begin{remark}
    It is easy to see that $V \subseteq V(I(V))$.
\end{remark}
\begin{theorem}[\textbf{\emph{Hilbert Nullstellensatz Theorem}}]
    Let $k$ be an algebraically closed field, $I \subseteq k[x_{1}, \cdots, x_{n}]$.
    Assume that $f\in k[x_{1}, \cdots, x_{n}]$ vanish at all points of $V(I)$.
    Then $f^{r}\in I$ for some $r > 0$.
    \label{Theorem 1.1}
\end{theorem}
\begin{corollary}
    Let $k$ be an algebraically closed field.
    Then there are some one-to-one correspondences
    \begin{equation*}
        \begin{split}
            \text{radical ideals of $k[x_{1}, \cdots, x_{n}]$} & \leftrightsquigarrow \text{affine varieties} \\ 
            \text{prime ideals of $k[x_{1}, \cdots, x_{n}]$} & \leftrightsquigarrow \text{irreducible affine varieties} \\ 
            \text{maximal ideals of $k[x_{1}, \cdots, x_{n}]$} & \leftrightsquigarrow \text{one-point affine varieties} 
        \end{split}
    \end{equation*}
    \label{Corollary 1.1}
\end{corollary}
\vspace{-\baselineskip} % 减少一行间距
\begin{definition}
    Let $V \subseteq \mathbb{A}_{k}^{n}$ be an affine variety, $I = I(V)$.
    Define the affine coordinate ring of $V$ to be $A_{V} := k[x_{1}, \cdots, x_{n}]/I$.
\end{definition}
\begin{corollary}
    Every finitely generated reduced $k$-algebra is the coordinate function ring of some affine variety.
    \label{Corollary 1.2}
\end{corollary}

\subsection{Regular maps}
\label{subsection:Affine_Variety_Regular_maps}

\begin{definition}
    Let $V, W$ be two affine varieties.
    Assume that $V \subseteq \mathbb{A}_{k}^{n}$ with coordinate $x_{1}, \cdots, x_{n}$and $W \subseteq \mathbb{A}_{k}^{m}$ with coordinate $y_{1}, \cdots, y_{m}$.
    A regular map from $V$ to $W$ is a map $\phi: V \rightarrow W$ satisfying there exist polynomials $f_{1}, \cdots, f_{m}\in k[x_{1}, \cdots, x_{n}]$ such that $\phi = (f_{i})_{1 \le i \le n}\big{|}_{V}$.
\end{definition}
\begin{remark}
    Given a regular map $\phi: V \rightarrow W$, there is a morphism $\phi^{\ast}: A_{W} \rightarrow A_{V}$ mapping $f$ to $f \circ \phi$.
\end{remark}
\begin{proposition}
    Let $V, W$ be two affine varieties.
    Assume that $V \subseteq \mathbb{A}_{k}^{n}$ with coordinate $x_{1}, \cdots, x_{n}$and $W \subseteq \mathbb{A}_{k}^{m}$ with coordinate $y_{1}, \cdots, y_{m}$.
    The correspondence $\phi \leftrightsquigarrow \phi^{\ast}$ is one-to-one.
    In fact, $\phi = (\phi^{\ast}(y_{1}), \cdots, \phi^{\ast}(y_{m}))$.
    \label{Proposition 1.1}
\end{proposition}
\begin{corollary}
    In particular, $\phi$ is isomorphic if and only if $\phi^{\ast}$ is isomorphic.
    \label{Corollary 1.3}
\end{corollary}
\begin{definition}
    Let $\phi: V \rightarrow W$ be a regular map of affine varieties.
    We say that $\phi$ is dominant if $\phi(V)$ is dense in $W$.
\end{definition}
\begin{proposition}
    Let $\phi: V \rightarrow W$ be a regular map of affine varieties.
    Then $\phi$ is dominant if and only if $\phi^{\ast}$ is injective.
    In addition, if $\phi^{\ast}$ is surjective, then $\phi$ is injective.
    \label{Proposition 1.2}
\end{proposition}
\begin{proof}
    Assume that $\phi$ is dominant.
    Suppose that $\phi^{\ast}$ is not injective, then there exist different $f,g\in A_{W}$ such that $f \circ \phi = g \circ \phi$.
    Since there exist $y\in W$ and an open neighbourhood $W \supseteq U\ni y$ such that $f\big{|}_{U} \neq g\big{|}_{U}$ everywhere in $U$.
    But $\phi(V)$ is dense in $W$, we get $x\in V$ such that $\phi(x)\in U$, contradiction!
    \par
    For the converse, assume that $\phi^{\ast}$ is injective.
    Suppose that $\phi$ is nor dominant.
    Then $\overline{\phi(V)} \subsetneq W$ so that $I(W) \subsetneq I(\overline{\phi(V)})$.
    Hence we can choose nonzero $f\in A_{W}$ s.t. $f$ vanish at all points of $\phi(V)$.
    Thus $\phi^{\ast}(f) = 0 = \phi^{\ast}(0)$ contradicting to $\phi^{\ast}$ is injective.
    \par
    Assume that $\phi^{\ast}$ is surjective.
    For any two points $x_{1}, x_{2}\in V$, there exists $F\in A_{V}$ such that $F(x_{1}) \neq F(x_{2})$.
    As $\phi^{\ast}$ is surjective, there exists $f\in A_{W}$ such that $F = f \circ \phi$.
    Thus $F(x_{1}) = f(\phi(x_{1}))$ and $F(x_{2}) = f(\phi(x_{2}))$ so that $\phi(x_{1}) \neq \phi(x_{2})$.
    Hence $\phi$ is injective.
\end{proof}

\section{Projective Variety}
\label{section:Projective_Variety}

\subsection{Zariski topology}
\label{subsection:Projective_Variety_Zariski_topology}

\begin{definition}
    Let $k$ be a field.
    Define the projective space over $k$ of dimension $n$ to be $\mathbb{P}_{k}^{n} := \{k^{\oplus n + 1} \setminus \{0\}\}/\sim$,
    where $x \sim y$ if and only if $x = ay$ for some $a\in k^{\times}$.
    Element in $\mathbb{P}_{k}^{n}$ is of the form $[a_{0}, a_{1}, \cdots, a_{n}]$.
\end{definition}
\begin{remark}
    To give algebraic in $\mathbb{P}_{k}^{n}$, we cannot directly use set of polynomials since for general polynomial $f$ and $x \sim y\in k^{\oplus n + 1} \setminus \{0\}$,
    $f(x) = 0 \iff f(y) = 0$ is not always correct.
    However, if $f$ is homogeneous, then $f(x) = a^{n + 1}f(y)$ for some $a\in k^{\times}$ so that $f(x) = 0 \iff f(y) = 0$.
\end{remark}
\begin{definition}
    Let $\mathbb{P}_{k}^{n}$ be projective space.
    Assume $S \subset k[x_{0}, x_{1}, \cdots, x_{n}]$ be a set of homogeneous polynomials.
    Define $V(S) := \{x\in \mathbb{P}_{k}^{n} \big{|} f(x) = 0, \forall f\in S\}$.
    Such a set is called an algebraic set.
\end{definition}
\begin{lemma}
    Let $\mathbb{P}_{k}^{n}$ be projective space.
    Then algebraic sets have following properties 
    \begin{itemize}
        \item For two subset $S_{1}, S_{2} \subset k[x_{0}, x_{1}, \cdots, x_{n}]$ of homogeneous polynomials,
        $V(S_{1}) \cup V(S_{2}) = V(S_{1}S_{2})$, where $S_{1}S_{2} = \{fg \big{|} f\in S_{1}, g\in S_{2}\}$.
        \item For any collection $\{S_{i}\}$, $\cap_{i} V(S_{i}) = V(\cup_{i} S_{i})$
    \end{itemize}
    \label{Lemma 2.1}
\end{lemma}
\vspace{-\baselineskip} % 减少一行间距
\begin{remark}
    With this lemma, we immediately have that algebraic sets give a topological structure on $\mathbb{P}_{k}^{n}$ as closed subsets, called the Zariski topology.
\end{remark}

\subsection{Projective variety}
\label{subsection:Projective_Variety_Projective_variety}

From now on, we would rename algebraic variety in projective space as projective variety.
\begin{definition}
    Let $\mathbb{P}_{k}^{n}$ be projective space, $V \subseteq \mathbb{A}_{k}^{n}$ a projective subvariety.
    Define the associated homogeneous ideal $I(V)$ of $V$ to be the ideal generated by $\{\text{homogeneous } f\in k[x_{0}, x_{1}, \cdots, x_{n}] \big{|} f(x) = 0, \forall x\in V\}$
\end{definition}
Similarly, by Theorem \ref{Theorem 1.1}, we get some correspondences.
\begin{corollary}
    Let $k$ be an algebraically closed field.
    Then there are some one-to-one correspondences
    \begin{equation*}
        \begin{split}
            \text{radical homogeneous ideals of $k[x_{1}, \cdots, x_{n}]$} & \leftrightsquigarrow \text{projective varieties} \\ 
            \text{prime homogeneous ideals of $k[x_{1}, \cdots, x_{n}]$} & \leftrightsquigarrow \text{irreducible projective varieties} \\ 
            \text{maximal homogeneous ideals of $k[x_{1}, \cdots, x_{n}]$} & \leftrightsquigarrow \text{one-point projective varieties} 
        \end{split}
    \end{equation*}
    \label{Corollary 2.1}
\end{corollary}
\vspace{-\baselineskip} % 减少一行间距
\begin{lemma}
    Let $\mathbb{P}_{k}^{n}$ be projective space with coordinate $x_{0}, x_{1}, \cdots, x_{n}$.
    Then for all $0 \le i \le n$, the open subset $D_{+}(x_{i}) := \{x_{i} \neq 0\}$ is homeomorphic to affine space $\mathbb{A}_{k}^{n}$ with Zariski topologies.
    \label{Lemma 2.2}
\end{lemma}
\vspace{-\baselineskip} % 减少一行间距
Hence, we can always embed affine variety in projective space.
In particular, we would get $\mathbb{P}_{k}^{0} \cong \mathbb{A}_{k}^{1}$.
Thus after that we would generally state propositions of projective varieties without explanation for affine case.
\par
In addition, the above lemma indicates that open subset in Zariski topology would also have some special properties.
So we here give it a formal definition.
\begin{definition}
    Let $k$ be a field.
    An open subset $U$ of projective (resp. affine) space is called a quasi-projective (resp. affine) variety.
\end{definition}

\subsection{Regular functions and regular maps}
\label{subsection:Projective_Variety_Regular_functions_and_regular_maps}

\begin{definition}
    Let $X$ be a (quasi-)projective subvariety of $\mathbb{P}_{k}^{n}$.
    A function $f: X \rightarrow k$ is said to be regular at $x\in X$ if there exist homogeneous polynomials $G, H$ of same degree such that $f = \frac{G}{H}$ in an open neighbourhood $U$ of $x$ and $H(p) \neq 0$ for all $p\in U$.
    \par
    We say that $f$ is a regular function if it is regular at all points of $X$.
    Denote $A(X)$ to be the ring of regular functions on $X$.
\end{definition}
\begin{remark}
    Regular functions are continuous map since locally they are defined as rational fractions.
\end{remark}
\begin{theorem}
    Let $k$ be an algebraically closed, $X$ irreducible (quasi-)affine subvariety of $\mathbb{A}_{k}^{n}$.
    Then a regular function $f: X \rightarrow k$ is defined by a polynomial i.e. regular function ring is just coordinate ring.
    \label{Theorem 2.1} 
\end{theorem}
\begin{proof}
    Take open covering $\{U_{i}\}$ such that $f\big{|}_{U_{i}}$ is defined as $\frac{G_{i}}{F_{i}}$, 
    where $G_{i}$ and $F_{i}$ are homogeneous polynomials of same degree.
    Hence $F_{i}G_{j} = F_{j}G_{i}$ on each $U_{i} \cap U_{j}$.
    Note that $U_{i} \cap U_{j}$ is nonempty open in $X$ which is irreducible, so we have that $X = \overline{U_{i} \cap U_{j}} \subseteq V(F_{i}G_{j} - F_{j}G_{i})$.
    \par
    Now $F_{i}G_{j} = F_{j}G_{i}$ on $X$ for all $i,j$.
    As $U_{i}$ cover $X$ and $H_{i}$ has no root in $U_{i}$, there is no common root of all $H_{i}$.
    Thus by Hilbert Nullstellensatz, there exist polynomials $a_{i}$ such that $\sum a_{i}H_{i} = 1$ on $X$.
    Take $g = \sum a_{j}G_{j}$, then 
    \begin{equation}
        H_{i}g = H_{i}\sum a_{j}G_{j} = \sum a_{j}H_{j}G_{i} = G_{i}
    \end{equation}
    get $\frac{G_{i}}{H_{i}} = g$ on $X$ for all $i$ so that $f$ is defined by $g$.
\end{proof}
\begin{remark}
    With this theorem, now we can identify regular function ring and coordinate ring of affine varieties.
    From now on, we would use same notation $A(X)$ for them.
    \par
    In addition, the irreducible condition is not required. 
    However, now we have not enough knowledge to prove it.
\end{remark}
\begin{definition}
    Let $X, Y$ be (quasi-)projective varieties.
    Wa say a map $\varphi: X \rightarrow Y$ is regular if 
    \begin{itemize}
        \item $varphi$ is continuous
        \item for all regular function $f: U \rightarrow k$ with $U \subseteq Y$ open,
        $f \circ \varphi: \varphi^{-1}(U) \rightarrow k$ is a regular function
    \end{itemize}
\end{definition}
\begin{example}
    For homogeneous polynomial $F_{0}, \cdots, F_{m}$ of same degree with no common root,
    we can define a regular map $\mathbb{P}_{k}^{n} \longrightarrow \mathbb{P}_{k}^{n}$ defined by
    \begin{equation}
        [x_{0}, x_{1}, \cdots, x_{n}] \longmapsto [F_{0}(x_{0}, \cdots, x_{n}), F_{1}(x_{0}, \cdots, x_{n}), \cdots, F_{m}(x_{0}, \cdot, x_{n})]
    \end{equation}
    In particular, if we take $F_{0} = x_{0}^{d}$, $F_{1} = x_{0}^{d - 1}x_{1}$, $\cdots$ and $F_{N} = x_{n}^{d}$ where $N = \binom{n + d}{d} - 1$,
    the corresponding map $v_{d}: \mathbb{P}_{k}^{n} \rightarrow \mathbb{P}_{k}^{N}$ is called the $d$th Veronese map.
\end{example}
\begin{lemma}
    Let $V, W$ be two affine varieties.
    Then the above definition of regular map from $V$ to $W$ is equivalent to our original definition.
    \label{Lemma 2.3}
\end{lemma}
\begin{reason}
    By Proposition \ref{Proposition 1.1}, it is clear that the two definitions are same.
\end{reason}
\begin{proposition}
    Let $X$ be a (quasi-)projective subvariety of $\mathbb{P}_{k}^{m}$, $\varphi: X \rightarrow \mathbb{P}_{k}^{m}$ a regular map.
    Then for every $x\in X$, there exists open neighbourhood $U$ of $x$ and homogeneous polynomials $F_{0}, \cdots, F_{m}$ such that $\varphi(y) = [F_{0}(y), \cdots, F_{n}(y)]$ for all $y\in U$.
    \label{Proposition 2.1}
\end{proposition}
\begin{proof}
    For $x\in X$, take affine open neighbourhood $D_{+}(x_{i})$.
    Without loss of generality, we may assume that $x\in D_{+}(x_{0}) \cong \mathbb{A}_{k}^{m}$ with coordinate $x_{1}, \cdots, x_{m}$.
    In addition, we can also take affine open neighbourhood of $\varphi(x)$ in $\mathbb{P}_{k}^{n}$, say $\mathbb{A}_{k}^{n}$ with coordinate $y_{1}, \cdots, y_{n}$.
    \par
    Now $U := X \cap D_{+}(x_{0}) \cap \varphi^{-1}(\mathbb{A}_{k}^{n})$ is an (quasi-)affine subvariety of $\mathbb{A}_{k}^{m}$.
    By Theorem \ref{Theorem 2.1}, $y_{i} \circ \varphi\big{|}_{U}: U \rightarrow \mathbb{A}_{k}^{n} \rightarrow k$ is defined by a polynomial $F_{i}$.
    Hence $\varphi\big{|}_{U} = [1, F_{1}, \cdots, F_{n}]$.
\end{proof}
\begin{example}
    Such local polynomials cannot glue up in general.
    Take $Z = V(xz - y^{2})$.
    Consider the following map $Z \rightarrow \mathbb{P}_{k}^{1}$ defined by 
    \begin{equation}
        [x, y, z] \longmapsto
        \left\{
            \begin{aligned}
               & [x, y] & x \neq 0 \text{ or } y \neq 0 \\
               & [y, z] & z \neq 0
            \end{aligned}
        \right.
    \end{equation}
\end{example}

\subsection{Rational maps}
\label{subsection:Projective_Variety_Rational_maps}

\begin{definition}
    Let $X, Y$ be (quasi-)projective varieties.
    Consider pairs $(U, f_{U})$ where $U \subset X$ is a dense open subset and $f_{U}: U \rightarrow Y$ is a regular map.
    We $(U, f_{U}) \sim (V, f_{V})$ if $f\big{|}_{U \cap V} = f_{V}\big{|}_{U \cap V}$.
    Such an equivalence class is called a rational map $X \dashrightarrow Y$.
    In particular, if $Y$ is $k$, then rational map $X \dashrightarrow k$ is called a rational function.
\end{definition}
\begin{remark}
    For any rational map $f: X \dashrightarrow Y$, there exists maximal open subset $U$ of $X$ such that $f\big{|}_{U}: U \dashrightarrow Y$ is a regular map,
    called the domain of definition of $f$.
\end{remark}
\begin{lemma}
    Let $X$ be an affine subvariety of $\mathbb{A}_{k}^{n}$.
    Then for any polynomial $f\in k[x_{1}, \cdots, x_{n}]$. the open set $D(f) \cap X$ is an affine subvariety of $\mathbb{A}_{k}^{n + 1}$,
    which is defined by $I(X)$ and $x_{n + 1}f - 1$, with coordinate ring $A(D(f) \cap X) = A(X)_{f}$.
    \label{Lemma 2.4}
\end{lemma}
\begin{remark}
    The proof is easy.
    Note that by this lemma, it is not necessary for us to define the notion of quasi-affine variety.
    Hence from now on, we would omit the prefix of quasi-affine variety.
    Also, for convenience, we mean (quasi-)projective variety when we say variety.
\end{remark}
\begin{proposition}
    Let $X$ be a variety.
    Then \\
    (a)$\{f: X \dashrightarrow k\}$ form a $k$-algebra, denoted by $K(X)$. \\
    (b)$\forall U \subset X$ open dense subset, we have that $K(X) = K(U)$. \\
    (c)If $X$ is irreducible, then $K(X)$ is a field. \\
    (d)If $X$ is affine and irreducible, then $K(X)$ is the fraction field of $A(X)$.
    \label{Proposition 2.2}
\end{proposition}
\begin{proof}
    (a), (b) are both obvious. Here we only prove (c) and (d).
    For (c), assume $f: X \dashrightarrow k$ is nonzero element in $K(X)$.
    Take its domain of definition $U$, then there exists open subset $U' \subseteq U$ such that $f\big{|}_{U'}$ is defnied by some polynomial $P$.
    Since $X$ is irreducible, $U'$ is dense so that $P: U' \rightarrow k$ also represents $f$.
    \par
    Consider the open subset $W = \{x\in X \big{|} P(x) \neq 0\}$.
    As $f$ is nonzero element, $W \cap U' \neq \varnothing$.
    Define $g: W \rightarrow k$ mapping $x$ to $\frac{1}{P(x)}$, which is a regular function and hence a regular map.
    Thus we get inverse of $f$ in $K(X)$.
    Conclude that $K(X)$ is a field.
    \par
    For (d), assume that $f: X \dashrightarrow k$ can be represented by $P: U \rightarrow k$.
    By definition of Zariski topology, we can take open subset $U' \subseteq U$ of the form $D(H)$ for some $H\in A(X)$.
    Hence by Lemma \ref{Lemma 2.4}, $f\in A(U') = A(X)_{H}$.
    Thus $K(X) \subseteq \cup_{H} A(X)_{H} = \fraction(A(X))$.
    The converse containment is obvious so that $K(X) = \fraction(A(X))$.
\end{proof}
\begin{remark}
    For irreducible variety $X$, there always exists some affine open subset $U$.
    Then by (c), $K(X) = K(U) = \fraction(A(U))$ is computable.
\end{remark}
\begin{example}
    Let $X \subseteq \mathbb{P}_{k}^{m}$ be a variety.
    Assume that homogeneous polynomials $F_{0}, \cdots, F_{n}\in k[x_{0}, \cdots, x_{m}]$ satisfy that $V(F_{0}, \cdots, F_{n})$ does not contain any irreducible component of $X$,
    then $X \setminus V(F_{0}, \cdots, F_{n})$ is dense in $X$ and we have a rational map $X \dashrightarrow \mathbb{P}_{k}^{n}$ defined by
    \begin{equation}
        x \longmapsto [F_{0}(x), \cdots, F_{n}(x)]
    \end{equation}
\end{example}
\vspace{-\baselineskip} % 减少一行间距
\begin{definition}
    Let $X, Y$ be varieties.
    We say a regular map $f: X \rightarrow Y$ is dominant if image of $f$ is dense in $Y$.
    \par
    We say a rational map $f: X \dashrightarrow Y$ is dominant if $f$ can be represented by some dominant regular map $f\big{|}_{U}: U \rightarrow Y$.
\end{definition}
\begin{remark}
    In general, we cannot directly define composition of two rational maps.
    But for a dominant rational map $f: X \dashrightarrow Y$ and any rational map $g: Y \dashrightarrow Z$,
    $g \circ f$ is a well-defined rational map.
    \par
    In addition, for a dominant rational map $f: X \dashrightarrow Y$, 
    it induces a canonical map $f^{\ast}: K(Y) \rightarrow K(X)$ mapping $g$ to $f \circ f$, which is a homomorphism of $k$-algebras.
\end{remark}
\begin{proposition}
    Assume that $X, Y$ are irreducible varieties.
    Then \\
    (1)There exists one-to-one correspondence 
    \begin{equation*}
        f: X \dashrightarrow Y \text{ dominant rational map} \leftrightsquigarrow K(Y) \hookrightarrow K(X) \text{ extension of fields}
    \end{equation*}
    (2)If there exists an isomorphism $\iota: K(Y) \overset{\sim}{\rightarrow} K(X)$,
    then there exists open dense subset $U \subseteq X$ and $V \subseteq Y$ such that $U \overset{\sim}{\rightarrow} V$ under the corresponding dominant rational map.
    \label{Proposition 2.3}
\end{proposition}

\subsection{Birational equivalence}
\label{subsection:Projective_Variety_Birational_equivalence}

\begin{definition}
    Let $X, Y$ be varieties.
    We say that $X, Y$ are birational if there exists open dense susbets $U \subseteq X$ and $V \subseteq Y$ such that $U \overset{\sim}{\rightarrow} V$.
\end{definition}
\begin{definition}
    Let $f: X \dashrightarrow Y$ be a dominant rational map.
    We say that $f$ is birational if there exists a dominant rational map $g: Y \dashrightarrow X$ such that $g \circ f = \identity_{X}$ and $f \circ g = \identity_{Y}$.
\end{definition}
\begin{corollary}
    Let $X, Y$ be varieties.
    Then $X$ and $Y$ are birational if and only if there exists a birational map $f: X \dashrightarrow Y$.
    \label{Corollary 2.2}
\end{corollary}
\begin{corollary}
    Let $X, Y$ be irreducible varieties.
    Then there exists one-to-one correspondence 
    \begin{equation*}
        f: X \dashrightarrow Y \text{ birational map} \leftrightsquigarrow \text{isomorphism from $K(Y)$ to $K(X)$ over $k$}
    \end{equation*}
    \label{Corollary 2.3}
\end{corollary}
\vspace{-\baselineskip} % 减少一行间距
\vspace{-\baselineskip} % 减少一行间距
\begin{proposition}
    Let $X$ be an irreducible variety.
    Then there exists a hypersurface in $\mathbb{A}_{k}^{n + 1}$ birational to $X$,
    where $n = \trdeg_{k}K(X)$.
    \label{Proposition 2.4}
\end{proposition}
\begin{proof}
    Let $X$ be an irreducible variety.
    Since the problem is local, we can assume that $X$ is affine.
    Then $A(X)$ is finitely generated $k$-algebra.
    By Noetherian Normalization Theorem, there exists transcendental basis $x_{1}, \cdots, x_{n}$ of $K(X)$ over $k$ such that $K(X)/k(x_{1}, \cdots, x_{n})$ is a finite extension.
    \par
    Since every finite extension of fields is a simple extension,
    there exists some $\alpha\in K(X)$ such that
    \begin{equation}
        \begin{split}
            K(X) & = k(x_{1}, \cdots, x_{n})[\alpha] \\
            & \cong k(x_{1}, \cdots, x_{n})[x_{n + 1}]/(f_{\alpha})
        \end{split}
    \end{equation}
    where $f_{\alpha}$ is the minimal polynomial of $\alpha$ in $k(x_{1}, \cdots, x_{n})$.
    Assume that $f_{\alpha} = x^{d} + P_{1}x^{d - 1} + \cdots + P_{d}$, where $P_{i} = \frac{A_{i}}{B_{i}}$ for some polynomial $A_{i}, B_{i}$.
    \par
    Take $B = \prod_{i} B_{i}$, then $f_{\alpha}$ can be viewed as an element in $k[x_{1}, \cdots, x_{n}]_{B}[x_{n + 1}] = k[x_{1}, \cdots, x_{n + 1}]_{B}$.
    Now $K(X) \cong \fraction(k[x_{1}, \cdots, x_{n + 1}]_{B}/(f_{\alpha}))$.
    Hence by Proposition \ref{Proposition 2.3}, $X$ is birational to the hypersurface $V(f_{\alpha}) \subset \mathbb{A}_{k}^{n + 1}$.
\end{proof}

\section{Blow Up}
\label{section:Blow_Up}

Blow up is a tool to resolute the singular point i.e. for variety $X$ with singularities,
by blowing up, we can find variety $Y$ with possibly less singularities and a regular map $Y \rightarrow X$.

\subsection{Segre embedding}
\label{subsection:Blow_Up_Segre_embedding}

\begin{definition}
    For projective space $\mathbb{P}_{k}^{m}$ with coordinate $x_{0}, \cdots, x_{m}$ and $\mathbb{P}_{k}^{n}$ with coordinate $y_{0}, \cdots, y_{n}$,
    there is a map $\mathbb{P}_{k}^{m} \times \mathbb{P}_{k}^{n} \rightarrow \mathbb{P}_{k}^{mn + m + n}$ defined by
    \begin{equation}
        ([x_{0}, \cdots, x_{m}], [y_{0}, \cdots, y_{n}]) \longmapsto [x_{i}y_{j}]_{i, j}
    \end{equation}
    called the Segre embedding.
    \par
    Set coordinate $z_{ij}$ for $\mathbb{P}_{k}^{mn + m + n}$.
    Then $[z_{ij}]$ is in the image of Segre embedding if and only if the following matrix is of rank $1$
    \begin{equation}
        \begin{pmatrix}
            z_{00} & \cdots & z_{0n} \\
            z_{10} & \cdots & z_{1n} \\
            \vdots & \ddots & \vdots \\
            z_{m0} & \cdots & z_{mn} 
        \end{pmatrix}
    \end{equation}
    which is equivalent to determinants of submatrix of the form
    $
    \begin{pmatrix}
        z_{ij} & z_{ij'} \\
        z_{i'j} & z_{i'j'} 
    \end{pmatrix}
    $
    all vanish.
    Thus the image of Segre embedding is the subvariety cut out by $\{z_{ij}z_{i'j'} - z_{i'j}z_{ij'} \big{|} \forall i, j, i', j'\}$.
\end{definition}
\begin{remark}
    The product topology and Zariski topology induced by Segre embedding on $\mathbb{P}_{k}^{m} \times \mathbb{P}_{k}^{n}$ are not same topology.
\end{remark}
\begin{definition}
    Let $f\in k[x_{1}, \cdots, x_{m}, y_{1}, \cdots, y_n]$ be a polynomial.
    We say that $f$ is bihomogenous of degree $(d, e)$ if $\forall u,v\in k$, we have that 
    \begin{equation}
        f(ux_{1}, \cdots, ux_{m}, vy_{1}, \cdots, vy_n) = u^{d}v^{e}f(x_{1}, \cdots, x_{m}, y_{1}, \cdots, y_{n})
    \end{equation}
    In fact, $f$ should be of the form $\sum_{i, j} a_{ij}x_{1}^{i_{1}}\cdots x_{m}^{i_{m}}y_{1}^{j_{1}}\cdots y_{n}^{j_{n}}$,
    where $i_{1} + \cdots + i_{m} = d$ and $j_{1} + \cdots + j_{n} = e$ for all $i, j$.
\end{definition}
\begin{proposition}
    Zariski closed subset of $\mathbb{P}_{k}^{m} \times \mathbb{P}_{k}^{n}$ is of the form $V(f_{1}, \cdots, f_{r})$ with $f_{i}$ bihomogenous.
    \label{Proposition 3.1}
\end{proposition}
\begin{proposition}
    Let $x, Y$ be varieties.
    Then \\
    (1)Projections $p_{1}: X \times Y \rightarrow X$ and $p_{2}: X \times Y \rightarrow Y$ are regular maps. \\
    (2)Given a map $f: Z \rightarrow X \times Y$ of varieties, $f$ is a regular map if and only if $p_{1} \circ f$ and $p_{2} \circ f$ are both regular maps.
    \label{Proposition 3.2}
\end{proposition}
\begin{definition}
    Let $f: X \rightarrow \mathbb{P}_{k}^{n}$ be a regular map.
    Define the graph $\Gamma_{f} \subseteq X \times \mathbb{P}_{k}^{n}$ of $f$ to be $\{(x, f(x)) \big{|} x\in X\}$.
\end{definition}
\begin{lemma}
    Let $f: X \rightarrow \mathbb{P}_{k}^{n}$ be a regular map.
    Then $\Gamma_{f}$ is closed.
    \label{Lemma 3.1}
\end{lemma}
\begin{reason}
    In fact, if consider regular map $X \times \mathbb{P}_{k}^{n} \rightarrow \mathbb{P}_{k}^{n} \times \mathbb{P}_{k}^{n}$ mapping $(x, y)$ to $(f(x), y)$.
    then $\Gamma_{f}$ is preimage of $\Delta_{\mathbb{P}_{k}^{n}}$ and hence closed.
\end{reason}

\subsection{Blow up of a point}
\label{subsection:Blow_Up_Blow_up_of_a_point}

Given a point in $\mathbb{P}_{k}^{n}$.
By coordinate transformation, without loss of generality, can assume the point is just origin $0 = [1, 0, \cdots, 0]$.
Consider the projection $\pi_{0}: \mathbb{P}_{k}^{n} \dashrightarrow \mathbb{P}_{k}^{n - 1}$ mapping $[x_{0}, \cdots, x_{n}]$ to $[x_{1}, \cdots, x_{n}]$,
which is a regular map outside $0$.
\par
Set $U = \mathbb{P}_{k}^{n} \setminus \{0\}$.
Consider the closure of $\Gamma_{\pi_{0}}$, called the blow up of $0$, denoted by $Bl_{0}\mathbb{P}_{k}^{n}$.
\begin{lemma}
    $Bl_{0}\mathbb{P}_{k}^{n} \subseteq \mathbb{P}_{k}^{n} \times \mathbb{P}_{k}^{n - 1}$ is defined by the bihomogenous polynomials $\{x_{i}y_{j} - x_{j}y_{i} \big{|} 1 \le i,j \le n\}$.
    \label{Lemma 3.2}
\end{lemma}
\begin{proof}
    Obviously, $Bl_{0}\mathbb{P}_{k}^{n} \subseteq V(\{x_{i}y_{j} - x_{j}y_{i}\})$.
    Want to show that there is an irreducible subsets covering $\{V_{i}\}$ of $V(\{x_{i}y_{j} - x_{j}y_{i}\})$ such that $\Gamma_{\pi_{0}} \cap V_{i}$ contains some nonempty open subset of $V_{i}$.
    \par
    Consider projection maps
    \begin{equation}
        \begin{tikzcd} 
            & V(\{x_{i}y_{j} - x_{j}y_{i}\}) \subseteq \mathbb{P}_{n}^{n} \times \mathbb{P}_{k}^{n - 1} \arrow[dr, "p_{2}"] \arrow[dl, "p_{1}" swap] & \\
            \mathbb{P}_{k}^{n} && \mathbb{P}_{k}^{n - 1}
        \end{tikzcd}
    \end{equation}
    For every $\beta = [b_{1}, \cdots, b_{n}]\in \mathbb{P}_{k}^{n - 1}$, 
    its preimage $p_{2}^{-1}(\beta)$ is the closed subset cut out by $x_{i}b_{j} - x_{j}b_{i}$.
    It is clear that $p_{2}^{-1}(\beta)$ is the line connecting $0$ and $[0, b_{1}, \cdots, b_{n}]$.
    \par
    For each $U_{i} = D_{+}(y_{i}) \subseteq \mathbb{P}_{k}^{n - 1}$, 
    we also have $p_{2}^{-1}(U_{i}) \cong \mathbb{P}_{k}^{1} \times U_{i}$ since $x_{i}y_{j} - x_{j}y_{i} = y_{i}(x_{i}\frac{y_{j}}{y_{i}} - x_{j})$ for all $j$ and hence 
    \begin{equation}
        \begin{split}
            p_{2}^{-1}(V_{i}) & = \{[x_{0}, x_{i}\frac{y_{1}}{y_{i}}, \cdots, x_{i}\frac{y_{n}}{y_{i}}]\} \\
            & \cong \{[x_{0}, x_{1}] \times (\frac{y_{1}}{y_{i}}, \cdots, \frac{y_{n}}{y_{i}})\} \\
            & = \mathbb{P}_{k}^{1} \times U_{i}
        \end{split}
    \end{equation}
    Now $p_{2}^{-1}(U_{i})$ is irreducible for all $i$.
    \par
    For each $\alpha = [a_{0}, \cdots, a_{n}] \neq 0$, we have that $p_{1}^{-1}(\alpha) = \alpha \times [a_{1}, \cdots, a_{n}]$.
    Hence 
    \begin{itemize}
        \item $p_{1}^{-1}(W_{i}) \overset{\sim}{\rightarrow} W_{i}$, where $W_{i} = D_{+}(x_{i}) \subseteq \mathbb{P}_{k}^{n}$
        \item $p_{1}^{-1}(0) = 0 \times \mathbb{P}_{k}^{n - 1}$
    \end{itemize}
    Note that $p_{1}^{-1}(W_{i}) \subseteq \Gamma_{\pi_{0}} \cap p_{2}^{-1}(U_{i})$ is open in $\mathbb{P}_{k}^{n} \times \mathbb{P}_{k}^{n - 1}$, done!
\end{proof}
\begin{remark}
    Above argument also shows that $p_{1}$ is an isomorphism over $\mathbb{P}_{k}^{n} \setminus \{0\}$.
\end{remark}
\begin{definition}
    Let $X$ be a subvariety of $\mathbb{P}_{k}^{n}$, $x\in X$.
    Define blow up of $X$ at $x$ to be $Bl_{x}X := \overline{p_{1}^{-1}(X \setminus \{x\})}$ with a commutative diagram
    \begin{equation}
        \begin{tikzcd}
            Bl_{x}X \arrow[r, hookrightarrow] \arrow[d, "p_{X}"] &
            Bl_{x}\mathbb{P}_{k}^{n} \arrow[d, "p_{1}"] \\
            X \arrow[r, hookrightarrow] &
            \mathbb{P}_{k}^{n}
        \end{tikzcd}
    \end{equation}
    Similarly, $p_{X}$ is an isomorphism over $X \setminus \{x\}$ and $p_{X}^{-1}(x) \subseteq x \times \mathbb{P}_{k}^{n - 1}$.
\end{definition}

\subsection{Resoluting singularities}
\label{subsection:Blow_Up_Resoluting_singularities}

\begin{example}
    (1)Take curve $C = V(x_{0}x_{1}^{2} - x_{2}^{2}(x_{0} - x_{2})) \subseteq \mathbb{P}_{k}^{2}$ with $0\in C$.
    Consider $U_{0} = D_{+}(x_{0})$.
    Then $C \cap U_{0} = V(x_{1/0}^{2} - x_{2/0}^{2} + x_{2/0}^{3})$.
    It is clear that $0$ is a singular point.
    By blow up at $0$, we would get $p_{1}^{-1}(0)$ is nonsingular.
    \\
    (2)For rational map $u: \mathbb{P}_{k}^{2} \rightarrow \mathbb{P}_{k}^{2}$ mapping $[x, y, z]$ to $[\frac{1}{x}, \frac{1}{y}, \frac{1}{z}] = [yz, zx, xy]$.
    Then $u$ is not regular at $p_{1} = [1, 0, 0]$, $p_{2} = [0, 1, 0]$ and $p_{3} = [0, 0, 1]$.
    By blowing up one by one, we would get $Bl_{p_{1}, p_{2}, p_{3}}\mathbb{P}_{k}^{2}$ such that there exists regular birational map $\phi$ making the following diagram commutative
    \begin{equation}
        \begin{tikzcd}
            Bl_{p_{1}, p_{2}, p_{3}}\mathbb{P}_{k}^{2} \arrow[dr, dashrightarrow, "\phi"] \arrow[d] & \\
            \mathbb{P}_{k}^{2} \arrow[r, dashrightarrow, "u"] &
            \mathbb{P}_{k}^{2}
        \end{tikzcd}
    \end{equation}
\end{example}
\begin{theorem}[\textbf{\emph{Hironaka}}]
    Let $k$ be an algebraically closed field of characteristic $0$.
    Then \\
    (1)For all variety $X$, there exists a series of blow-ups 
    \begin{equation}
        X_{n} \longrightarrow X_{n - 1} \longrightarrow \cdots \longrightarrow X_{1} \longrightarrow X_{0} = X
    \end{equation}
    such that $X_{n}$ is smooth. \\
    (2)For all rational map $f: X \dashrightarrow Y$, there exists a series of blow-ups 
    \begin{equation}
        \pi: X_{n} \longrightarrow X_{n - 1} \longrightarrow \cdots \longrightarrow X_{1} \longrightarrow X_{0} = X
    \end{equation}
    such that $f \circ \pi: X_{n} \rightarrow Y$ is regular.
    \label{Theorem 3.1}
\end{theorem}
For the second example above, we can say more things about birational maps of $\mathbb{P}_{k}^{2}$.
Set $Cr(\mathbb{P}_{k}^{2}) := \{f: \mathbb{P}_{k}^{2} \dashrightarrow \mathbb{P}_{k}^{2} \text{ birational}\}$, called Cremona group
By Corollary \ref{Corollary 2.3}, there is a one-to-one correspondence between $Cr(\mathbb{P}_{k}^{2})$ and $\aut(K(\mathbb{P}_{k}^{2})/k)$, 
where $K(\mathbb{P}_{k}^{2}) = K(\mathbb{A}_{k}^{2}) = k(x, y)$.
\par
Then it is clear that $u\in Cr(\mathbb{P}_{k}^{2})$ is an involution and $SL_{k}(3) \subseteq Cr(\mathbb{P}_{k}^{2})$.
In fact, by the following Theorem, $Cr(\mathbb{P}_{k}^{2})$ is just generated by these elements.
\begin{theorem}[\textbf{\emph{Noether-Castelnuovo}}]
    Any birational map $\varphi\in Cr(\mathbb{P}_{k}^{2})$ is of the form $M_{1} \circ u \circ M_{2} \circ \cdots M_{n} \circ u^{t}$,
    where $M_{i}\in SL_{k}(3)$ and $t\in \{0, 1\}$.
    \label{Theorem 3.2}
\end{theorem}

\subsection{Images of regular maps}
\label{subsection:Blow_Up_Images_of_regular_maps}

\begin{theorem}
    Assume that $X$ is a variety and $Y$ is a projective variety.
    Then the projection $p: X \times Y \rightarrow X$ is a closed map.
    \label{Theorem 3.3}
\end{theorem}
\begin{proof}
    Recall that resultant of polynomials $F = a_{0} + a_{1}x + \cdots + a_{s}x^{s}$ and $G = b_{0} + b_{1}x + \cdots + b_{t}x^{t}$ is a matrix of degree $s + t$
    \begin{equation}
        Res(F, G) = 
        \begin{pmatrix}
            a_{0} & a_{1} & \cdots & a_{s} &&& \\
            & a_{0} & a_{1} & \cdots & a_{s} && \\
            &&&& \ddots && \\
            &&& a_{0} & a_{1} & \cdots & a_{s} \\
            b_{0} & b_{1} & \cdots & b_{t} &&& \\
            & b_{0} & b_{1} & \cdots & b_{t} && \\
            &&&& \ddots && \\
            &&& b_{0} & b_{1} & \cdots & b_{t}
        \end{pmatrix}
    \end{equation}
    \par
    Step 1: Take $Y = \mathbb{P}_{k}^{1}$ and $X = \mathbb{P}_{k}^{m}$, want to show that $\pi: \mathbb{P}_{k}^{m} \times \mathbb{P}_{k}^{1} \rightarrow \mathbb{P}_{k}^{m}$ is closed.
    Let $Z \subseteq \mathbb{P}_{k}^{m} \times \mathbb{P}_{k}^{1}$ be a closed subset, defined by bihomogenous polynomials $F_{i}(x_{0}, \cdots, x_{m}, y_{0}, y_{1})$ for $1 \le i \le h$. 
    Define $\overline{F_{i}} = F_{i}(x_{0}, \cdots, x_{m}, 1, Y)\in k[x_{0}, \cdots, x_{m}][Y]$.
    Claim that $\pi(Z) \subseteq \mathbb{P}_{k}^{m}$ is the closed subset defined by $Res(\overline{F_{i}}, \overline{F_{j}})\in k[x_{0}, \cdots, x_{m}]$ for $1 \le i,j \le h$.
    \par
    For $\alpha\in \pi(Z)$, there exists $[\beta_{0}, \beta_{1}]\in \mathbb{P}_{k}^{1}$ such that $\alpha \times [\beta_{0}, \beta_{1}]\in Z$ so that $F_{i}(\alpha, \beta_{0}, \beta_{1}) = 0$ for all $i$.
    If $\beta_{0} \neq 0$, then replace $[\beta_{0}, \beta_{1}]$ by $[1, \frac{\beta_{1}}{\beta_{0}}]$ and hence $\frac{\beta_{1}}{\beta_{0}}$ is common root of all $\overline{F_{i}}(\alpha, Y)$.
    Thus $Res(\overline{F_{i}}, \overline{F_{j}}) = 0$ for all $i,j$.
    \par
    If $\beta_{0} = 0$, then leading coefficients of $\overline{F_{i}}(\alpha, Y)$ are all $0$ and hence $Res(\overline{F_{i}}, \overline{F_{j}})(\alpha) = 0$ for all $i,j$. 
    Thus we get $\pi(Z) \subseteq V(Res(\overline{F_{i}}, \overline{F_{j}}))$.
    \par
    Conversely, assume that all resultants vanish at $\alpha\in \mathbb{P}_{k}^{m}$.
    We need to show that $\alpha\in \pi(Z)$.
    If $\alpha$ is common root of all the leading coefficients of $\overline{F_{i}}$,
    then obviously $\alpha \times [0, 1]\in Z$ and hence $\alpha\ni \pi(Z)$.
    \par
    Suppose that there exists some $F\in I(Z)$ whose leading coefficient does not vanish at $\alpha$.
    Note that $\overline{F}(\alpha, 1, Y)$ has only finitely many roots, denoted by $\gamma_{1}, \cdots, \gamma_{N}$.
    Since $Res(\overline{F}, \overline{G})(\alpha) = 0$ and leading coefficient of $\overline{F}(\alpha, Y)$ is nonzero,
    we have that $G(\alpha, 1, Y)$ vanish at least one point in $\gamma_{1}, \cdots, \gamma_{N}$.
    \par
    Suppose that all $\alpha \times [1, \gamma_{i}]$ are not in $Z$.
    Then there exist $G_{i}$ such that $G(\alpha, 1, \gamma_{i}) \neq 0$.
    Multiplied by some power of $y$ and $x_{j}$, where $\alpha\in D_{+}(x_{j})$,
    we may assume that all $G_{i}$ are of same bidegree.
    \par
    Then for all $a_{i}$, bihomogenous polynomial $\sum_{i} a_{i}G_{i}\in I(Z)$ has some root in $\{\gamma_{1}, \cdots, \gamma_{N}\}$.
    Consider linear map $\phi: k^{N} \rightarrow k^{N}$ defined by 
    \begin{equation}
        (a_{1}, \cdots, a_{n}) \longmapsto (\sum_{i} a_{i}G_{i}(\alpha, 1, \gamma_{1}), \cdots, \sum_{i} a_{i}G_{i}(\alpha, 1, \gamma_{N}))
    \end{equation}
    Note that $\im \phi \subseteq \cup_{i} H_{i}$ where $H_{i} = \{a_{i} = 0\}$.
    While $\im \phi$ is vector space over infinite field $k$, 
    it cannot be covered by finitely many proper subspaces so that $\im \phi \subseteq H_{i}$ for some $i$,
    while implies that all $G_{j}(\alpha, 1, \gamma_{i}) = 0$.
    In particular, we get that $G_{i}(\alpha, 1, \gamma_{i}) = 0$, contradiciton!
    \par
    Step 2: For general variety $X$, want to show that $X \times \mathbb{P}_{k}^{1} \rightarrow X$ is closed.
    In this case, consider a commutative diagram
    \begin{equation}
        \begin{tikzcd}
            X \times \mathbb{P}_{k}^{1} \arrow[r, "\pi_{X}"] \arrow[d, hookrightarrow] &
            X \arrow[d, hookrightarrow] \\
            \mathbb{P}_{k}^{m} \times \mathbb{P}_{k}^{1} \arrow[r, "\pi"] &
            \mathbb{P}_{k}^{m}
        \end{tikzcd}
    \end{equation}
    Let $Z$ be a closed subset.
    Then there exists closed subset $Z' \subseteq \mathbb{P}_{k}^{m}$ such that $z' \cap X = Z$.
    Clearly, $\pi_{X}(Z) = \pi(Z') \cap X$ is still closed in $X$.
    \par
    Step 3: For general variety $X$, want to show that $X \times \mathbb{P}_{k}^{1} \rightarrow X$ is closed.
    Recall that for all $x\in \mathbb{P}_{k}^{n}$, we can blow up at $x$.
    For all standard affine open subset $U_{i} \subseteq \mathbb{P}_{k}^{n - 1}$, we have a commutative diagram
    \begin{equation}
        \begin{tikzcd}
            U_{i} \times \mathbb{P}_{k}^{1} \cong q^{-1}(U_{i}) \arrow[r, hookrightarrow] \arrow[d] &
            Bl_{x}\mathbb{P}_{k}^{n} \arrow[r, "p"] \arrow[d, "q"] &
            \mathbb{P}_{k}^{n} \\
            U_{i} \arrow[r, hookrightarrow] &
            \mathbb{P}_{k}^{n - 1} & 
        \end{tikzcd}
    \end{equation}
    Taking product by $X$ over $k$, we get a commutative diagram
    \begin{equation}
        \begin{tikzcd}
            X \times U_{i} \times \mathbb{P}_{k}^{1} \arrow[r, hookrightarrow] \arrow[d] &
            X \times Bl_{x}\mathbb{P}_{k}^{n} \arrow[r, "\identity \times p"] \arrow[d, "\identity \times q"] &
            X \times \mathbb{P}_{k}^{n} \arrow[d, "\pi_{X}"] \\
            X \times U_{i} \arrow[r, hookrightarrow] &
            X \times \mathbb{P}_{k}^{n - 1} \arrow[r] & 
            X
        \end{tikzcd}
    \end{equation}
    For all closed subset $Z \subseteq \mathbb{P}_{k}^{n}$, 
    set $\widetilde{Z} = (\identity \times p)^{-1}(Z)$ which is a closed subset of $X \times Bl_{x}\mathbb{P}_{k}^{n}$.
    Then $Z_{i} := \widetilde{Z} \cap (\identity \times q)^{-1}(X \times U_{i})$ is closed in $(\identity \times q)^{-1}(X \times U_{i})$.
    \par
    By step 1, $X \times U_{i} \times \mathbb{P}_{k}^{1} \rightarrow X \times U_{i}$ is closed so that $(\identity \times q)(Z_{i})$ is closed in $X \times U_{i}$ for all $i$.
    Thus $(\identity \times q)(\widetilde{Z})$ is closed in $X \times \mathbb{P}_{k}^{n - 1}$.
    By induction on $n$, $X \times \mathbb{P}_{k}^{n - 1}$ is closed so that $\pi_{X}(Z)$ is closed.
    \par
    Step 4: For general variety $X$ and projective $Y$, want to show that $X \times Y \rightarrow X$ is closed.
    Since $Y$ is projective, we can embed it into $\mathbb{P}_{k}^{n}$ for some $n$.
    Consider the following commutative diagram
    \begin{equation}
        \begin{tikzcd}
            X \times \mathbb{P}_{k}^{n} \arrow[r, "\pi_{X}"] \arrow[d, hookrightarrow] &
            X \arrow[d, hookrightarrow] \\
            \mathbb{P}_{k}^{m} \times \mathbb{P}_{k}^{n} \arrow[r, "\pi"] &
            \mathbb{P}_{k}^{m}
        \end{tikzcd}
    \end{equation}
    By argument similar to step 2, done.
\end{proof}
\begin{corollary}
    Let $Y$ be a projective variety.
    Then any regular map $f: Y \rightarrow Z$ is closed.
    \label{Corollary 3.1}
\end{corollary}
\begin{reason}
    For all closed subset $W \subseteq Y$, $\Gamma_{f} \cap W \times Z \hookrightarrow Y \times Z \rightarrow Z$ has closed image.
\end{reason}
\begin{corollary}
    Any regular function on a connected projective variety $X$ is constant.
    \label{Corollary 3.2}
\end{corollary}
\begin{proof}
    Let $f: X \rightarrow k = \mathbb{A}_{k}^{1}$ be a regular function.
    Then $\widetilde{f}: X \rightarrow \mathbb{A}_{k}^{1} \subseteq \mathbb{P}_{k}^{1}$ is a regular map.
    Hence $\widetilde{f}$ is closed so that $\widetilde{f}(X)$ is closed in $\mathbb{P}_{k}^{1}$.
    Note that $\widetilde{f}(X) \neq \mathbb{P}_{k}^{1}$ since $\mathbb{A}_{k}^{1} \hookrightarrow \mathbb{P}_{k}^{1}$ is not surjective.
    Thus $\widetilde{f}(X)$ is a finite set and by connectivity, $f$ is constant.
\end{proof}
\begin{corollary}
    Let $X \subseteq \mathbb{P}_{k}^{n}$ be a closed subvariety.
    Assume that $X$ is connected and $X$ is not a point.
    Then for all hypersurface $H \subseteq \mathbb{P}_{k}^{n}$, $X$ meets $H$.
    \label{Corollary 3.3}
\end{corollary}
\begin{proof}
    We say that $H = V(F)$ where $F$ is a homogeneous polynomial of degree $d$.
    Consider $d$th Veronese map $V_{d}: \mathbb{P}_{k}^{n} \hookrightarrow \mathbb{P}_{k}^{\binom{n + d}{d} - 1}$.
    Denote $N = \binom{n + d}{d} - 1$.
    Suppose that $F = \sum_{I} a_{i}x^{I}$, where $I$ varies $(a_{0}, a_{1}, \cdots, a_{n})$ satisfying that $\sum a_{i} = d$ and $x_{I} = \prod_{i} x_{i}^{a_{i}}$.
    Take $\widetilde{H}$ to be the hypersurface in $\mathbb{P}_{k}^{N}$ defined by $\sum_{I} a_{I}x_{I}$.
    Then $v_{d}^{-1}(\mathbb{P}_{k}^{N} \setminus \widetilde{H}) = \mathbb{P}_{k}^{n} \setminus H$.
    \par
    Suppose that $X \cap H = \varnothing$, then $X \subseteq \mathbb{P}_{k}^{n} \setminus H$.
    Now for all homogeneous polynomial $G$ of degree $d$, $\frac{G}{F}\big{|}_{X}$ defines a regular function on $X$.
    As $X$ is connected projective variety, by Corollary \ref{Corollary 3.2}, $\frac{G}{F}\big{|}_{X}$ is constant.
    Note that $\frac{G}{F}\big{|}_{X}$ can also be viewed as a regular function on $v_{d}(X)$.
    Taking $G$ varies $x^{I}$ for all $I$, it is clear that $v_{d}(X)$ is a point.
    While $v_{d}$ is injective, contradiction!
\end{proof}

\section{Dimension Theory}
\label{section:Dimension_Theory}
 
\subsection{Dimension}
\label{subsection:Dimension_Theory_Dimension}

\begin{definition}
    Let $X$ be a topological space.
    The dimension of $X$ denoted by $\dim X$ is the supremum of $n$ such that there exists a chain of irreducible closed sbusets of length $n$.
\end{definition}
\begin{proposition}
    Let $X$ be a topological space, $Y \subseteq X$ subspace with induced topology.
    Then we have that \\
    (1)$\dim Y \le \dim X$ \\
    (2)If $X$ is irreducible of finite dimension and $Y$ is proper subset, then $\dim Y < \dim X$. \\
    (3)If $X = \cup_{i} X_{i}$ is union of finitely many closed subsets, where $X_{i}$ are closed, then $\dim X = \max_{i} \{\dim X_{i}\}$.
    \label{Proposition 4.1}
\end{proposition}
\begin{corollary}
    Let $X$ be a variety.
    Assume that $\dim X = 0$, then $X$ is union of finitely many points.
    \label{Corollary 4.1}
\end{corollary}
\begin{reason}
    Since $X$ can be written as union of finitely many irreducible components,
    by previous proposition, we would get each component of $X$ is of dimension $0$ and hence is one-point set.
\end{reason}
\begin{remark}
    If each irreducible component of $X$ is of dimension $n$, 
    then we say that $X$ is pure of dimension $n$ (or say that $X$ is equidimensional of dimension $n$).
    \par
    Given $x\in X$, define the dimension of $X$ at $x$ to be $\dim_{X} x := \sup \{n \big{|} \exists \{x\} \subsetneq X_{1} \subsetneq \cdots \subsetneq X_{n} \text{ chain of irreducible closed subsets}\}$.
    Clearly, $\dim X \ge \dim_{X} x$.
\end{remark}
\begin{proposition}
    Let $X$ be an irreducible affine variety, $A_{X}$ coordinate ring.
    Then $\dim X = \dim A_{X}$.
    \label{Proposition 4.2}
\end{proposition}
\begin{reason}
    Immediately comes from the one-to-one correspondence between irreducible subvarieties and prime ideals.
\end{reason}
\begin{theorem}
    Let $A$ be an integral domain over field $k$ (not necessarily algebraically closed).
    Then $\dim A = \trdeg(\fraction(A)/k)$.
    \label{Theorem 4.1}
\end{theorem}
\begin{corollary}
    $\dim \mathbb{A}_{k}^{n} = n$.
    \label{Corollary 4.2}
\end{corollary}
\begin{theorem}
    Let $A$ be an integral domain of finite type over field $k$ (not necessarily algebraically closed).
    Then for all prime ideal $\mathfrak{p}$ of $A$, we have that $\height (\mathfrak{p}) + \dim A/\mathfrak{p} = \dim A$.
    \label{Theorem 4.3}
\end{theorem}
\begin{corollary}
    Let $X$ be an irreducible variety.
    Then for all irreducible subvariety $Y \subseteq X$, $\dim Y + \codim Y = \dim X$.
    In particular, $\dim_{X} x = \dim X$ for all $x\in X$.
    \label{Corollary 4.3}
\end{corollary}
\begin{proposition}
    Let $X$ be a variety.
    Then $\dim X$ is finite and for all open dense subset $U \subseteq X$, we have that $\dim U = \dim X$.
    \label{Proposition 4.3}
\end{proposition}
\begin{example}
    (1)$\dim \mathbb{P}_{k}^{n} = n$ since $D_{+}(x_{i}) \cong \mathbb{A}_{k}^{n}$ is an open dense subset. \\
    (2)$\dim (\mathbb{P}_{k}^{n} \times \mathbb{P}_{k}^{m}) = n + m$. \\
    (3)Assume that $X$ is irreducible, then $\dim X = \trdeg(K(X)/k)$.
    Suffice to take affine open dense subset $U \subseteq X$, then
    \begin{equation}
        \dim X = \dim U = \trdeg(K(U)/k) = \trdeg(K(X)/k)
    \end{equation}
    (4)Let $f: X \dashrightarrow Y$ be a dominant rational map corresponding to a field extension $K(Y) \hookrightarrow K(X)$.
    Hence $\dim Y \le \dim X$.
\end{example}

\subsection{Dimension and defining equations}
\label{subsection:Dimension_Theory_Dimension_and_defining_equations}

\begin{theorem}[\textbf{\emph{Krull's Principal Ideal Theorem}}]
    Let $A$ be a noetherian domain, $f\in A$ nonzero.
    Then all minimal prime ideals in $V(f)$ are of height $1$.
    \label{Theorem 4.3}
\end{theorem}
\begin{remark}
    For noetherian ring $A$ and $f\in A$, we also has Krull's Principal Ideal Theorem that all minimal prime ideals in $V(f)$ are of height at most $1$ and take equality if $f$ is not zero-divisor.
\end{remark}
\begin{theorem}
    Let $X \subseteq \mathbb{P}_{k}^{n}$ be a quasi-projective subvariety, $F_{1}, \cdots, F_{r}$ homogeneous polynomials.
    Then we have that \\
    (1)If $X$ is of pure dimension $n$ and $X \cap V(F_{i})$ is nonempty, 
    then each irreducible component of $X \cap V(F_{i})$ is of dimension $\ge n - r$. \\
    (2)If $X$ is closed of dimension $n$ and $r \le n$, then $X \cap V(F_{i}) \neq \varnothing$.
    \label{Theorem 4.4} 
\end{theorem}
\begin{remark}
    For (2), recall that by Corollary \ref{Corollary 3.3}, 
    for projective $X$ which is not one-point set, $X \cap V(F) \neq \varnothing$.
    This is just the $r = 1$ case.
\end{remark}
\begin{corollary}
    Hypersurfaces $V(F)$ of $\mathbb{A}_{k}^{n}$ (resp. $\mathbb{P}_{k}^{n}$) are exactly the closed subvarieties of pure dimension $n - 1$.
    \label{Corollary 4.4}
\end{corollary}
\begin{reason}
    Assume that $F = \prod_{i} F_{i}$ is the irreducible factorization.
    Then $V(F) = \cup_{i} V(F_{i})$ and by Proposition \ref{Proposition 4.1}, 
    $\dim V(F) = \max_{i} \{\dim V(F_{i})\} = \max_{i} \{n - 1\} = n - 1$.
    \par
    Conversely, assume that $Y \subseteq \mathbb{A}_{k}^{n}$ is an irreducible closed subvariety of dimension $n - 1$.
    Take some $F = \prod_{i} F_{i} \neq 0\in I(Y)$.
    Then $Y \subseteq V(F) = \cup_{i} V(F_{i})$.
    As $Y$ is irreducible, we may assume that $Y \subseteq V(F_{1})$ which is also of dimension $n - 1$ so that $Y = V(F_{1})$.
\end{reason}
\begin{example}
    Let $\mathbb{P}_{k}^{3}$ be projective space with coordinate $x, y, z, w$, 
    $F_{1} = xw - yz$, $F_{2} = xz - y^2$ and $F_{3} = yw - z^2$.
    Then $\dim V(F_{1}, F_{2}, F_{3}) = 1 > 3 - 3 = 0$.
    This is because $F_{3}$ is zero-divisor in $k[x, y, z, w]/(F_{1}, F_{2})$.
\end{example}

\subsection{Generically finite morphism}
\label{subsection:Dimension_Theory_Generically_finite_morphism}

From now on, by a morphism of varieties, we mean a regular map.
\begin{definition}
    Let $f: X \rightarrow Y$ be a morphism of varieties.
    Assume $\mathcal{P}$ is a property of fibers.
    We say that a general fiber of $f$ satisfies property $\mathcal{P}$ if there exists open dense subset $U \subseteq Y$ such that for all $y\in U$,
    fiber $f^{-1}(y)$ satisfies property $\mathcal{P}$.
\end{definition}
\begin{definition}
    Let $f: X \rightarrow Y$ be a morphism of varieties, $\characteristic k = 0$.
    Assume $\mathcal{P}$ is a property of fibers.
    We say that a very general fiber of $f$ satisfies property $\mathcal{P}$ if there exist countably many proper closed subset $V_{i} \subseteq Y$ such that for all $y\in \cup_{i} V_{i}$,
    fiber $f^{-1}(y)$ satisfies property $\mathcal{P}$.
\end{definition}
\begin{lemma}
    Let $X, Y$ be irreducible varieties, $f: X \rightarrow Y$ a dominant morphism.
    Assume that $U$ is an open dense subset of $Y$ and $V \subseteq f^{-1}(U)$ is an open dense subset of $X$.
    Then a general fiber of $f$ is finite if and only if a general fiber of $f\big{|}_{V}$ is finite.
    \label{Lemma 4.1}
\end{lemma}
\begin{proof}
    The "$\Rightarrow$" side is trivial since for all $y\in U$, $f\big{|}_{V}^{-1}(y) \subseteq f^{-1}(y)$.
    For the converse, since a finite fiber of $f\big{|}_{V}$ is finite, $\dim X = \dim V = \dim U = \dim Y$.
    Take $W = V^{c}$, then $W$ is a proper subset of $X$ and hence $\dim W < dim X$.
    As $f\big{|}_{W}: W \rightarrow \overline{f(W)}$ is dominant, $\dim \overline{f(W)} \le \dim W < \dim X = \dim Y$ so that $\overline{f(W)}$ is a proper subset.
    Now take $\overline{f(W)}^{c} \cap U$, we would see that for all $y\in \overline{f(W)}^{c} \cap U$, $f^{-1}(y) \subseteq V$, done!
\end{proof}
\begin{theorem}
    Let $X, Y$ be irreducible varieties, $f: X \rightarrow Y$ a dominant morphism.
    Then a general fiber of $f$ is finite if and only if the corresponding field extension $f^{\ast}: K(Y) \hookrightarrow K(X)$ is a finite extension.
    \label{Theorem 4.5}
\end{theorem}
\begin{remark}
    In this case, $\dim X = \dim Y$ and for general $y$, $\sharp f^{-1}(y) \le [K{X} : K(Y)]$ taking equality when $\characteristic k = 0$.
\end{remark}
\begin{proof}
    Firstly prove for affine case.
    Assume $X \subseteq \mathbb{A}_{k}^{n}$ and $Y \subseteq \mathbb{A}_{k}^{m}$.
    Consider graph $\Gamma_{f}$ of $f$, there is a commutative diagram
    \begin{equation}
        \begin{tikzcd}
            \Gamma_{f} \arrow[rr, hookrightarrow] \arrow[dr, "\sim"] &&
            X \times Y \arrow[rr, hookrightarrow] \arrow[dr] \arrow[dl] &&
            \mathbb{A}_{k}^{n + m} \arrow[dr] & \\
            & X \arrow[rr, "f"] &&
            Y \arrow[rr, hookrightarrow] &&
            \mathbb{A}_{k}^{m}
        \end{tikzcd}
    \end{equation}
    Replacing $X$ and $f$ by $\Gamma_{f}$ and $\Gamma_{f} \overset{\sim}{\rightarrow} X \overset{f}{\rightarrow} Y$ respectively,
    then the new $f$ is still a morphism induced by the canonical projection $\mathbb{A}_{k}^{n + m} \rightarrow \mathbb{A}_{k}^{m}$.
    \par
    Induct on $n$. 
    For $n = 1$, let the coordination of $\mathbb{A}_{k}^{1 + m}$ be $x, y_{1}, \cdots, y_{m}$.
    Then $A(X)$ is generated by $\overline{x}$ over $A(Y)$.
    If $x$ is algebraic over $K(Y)$, then assume $G(T)\in K(Y)[T]$ is the minimal polynomial of $x$.
    Write $G(T) = T^{d} + a_{1}T^{d - 1} + \cdots + a_{d}$, where $a_{i}\in K(Y)$.
    Multiplied by product of dominators, we get $\widetilde{G}(T) = b_{0}T^{d} + b_{1}T^{d - 1} + b_{d}$, where $b_{i}\in A(Y)$ and $b_{0} \neq 0$.
    \par
    For all $y\in Y$, as soon as $b_{0}(y) \neq 0$, then for all $(\alpha, y)\in f^{-1}(y)$,
    $\alpha$ would be zero of the equation $\widetilde{G}(T, y)$ hence $f^{-1}(y)$ is a finite set with $\sharp f^{-1}(y) \le d = [K(X) : K(Y)]$.
    Moreover, if $K(X)$ is separable over $K(Y)$, then the equality holds.
    \par
    If $x$ is transcendental over $K(Y)$, then for all $F\in I(X)$,
    written as $b_{0}(y_{i})x^{d} + b_{1}(y_{i})x^{d - 1} + \cdots + b_{d}(y_{i})$,
    we have that $b_{j}(y) = 0$ for all $y\in Y$ so that $b_{j}(y_{I})\in I(Y)$.
    Thus for all $y\in Y$ and all $a\in k$, $F(a, y) = 0$.
    Conclude that $\mathbb{A}_{k}^{1} \times \{y\} \subseteq X$ and hence $f^{-1}(y) = \mathbb{A}_{k}^{1} \times \{y\}$.
    \par
    For $n > 1$, consider following commutative diagram
    \begin{equation}
        \begin{tikzcd}
            \mathbb{A}_{k}^{n + m} \arrow[r] &
            \mathbb{A}_{k}^{n - 1 + m} \arrow[r] &
            \mathbb{A}_{k}^{n - 2 + m} \arrow[r] &
            \cdots \arrow[r] &
            \mathbb{A}^{m} \\
            X = Y_{0} \arrow[u, hookrightarrow] \arrow[r] &
            Y_{1} \arrow[u, hookrightarrow] \arrow[r] &
            Y_{2} \arrow[u, hookrightarrow] \arrow[r] &
            \cdots \arrow[r] &
            Y_{n} = Y \arrow[u, hookrightarrow]
        \end{tikzcd}
    \end{equation}
    By induction, we successfully prove the affine case.
    \par
    For general case, take affine open dense subset $U \subseteq Y$ and affine open dense subset $V \subseteq f^{-1}(U)$.
    Then $f\big{|}_{V}: V \rightarrow U$ is a dominant morphism between affine varieties.
    Hence a general fiber of $f\big{|}_{V}$ is finite if and only if $K(U) = K(Y)$ is finite over $K(V) = K(X)$.
    Then by Lemma \ref{Lemma 4.1}, done!
\end{proof}
\begin{remark}
    When we replace $X$ by $\Gamma_{f}$, we in fact transfer imformation about $f$ into $I(X)$ while the new $f$ becomes trivial.
\end{remark}
\begin{example}
    For the case that $\characteristic k = p$,
    take morphism $f: X = \mathbb{A}_{k}^{1} \rightarrow Y = \mathbb{A}_{k}^{1}$ sending $x$ to $x^{p}$.
    Then for all $\alpha\in \mathbb{A}_{k}^{1}$, $f^{-1}(\alpha) = \alpha^{\frac{1}{p}}$ is a unique point while $[K(X) : K(Y)] = p$.
\end{example}
\begin{corollary}
    Let $X$ be an irreducible variety of dimension $n$.
    Then there exists a dominant morphism $f: X \rightarrow \mathbb{P}_{k}^{n}$ with all fibers finite.
    \label{Corollary 4.5}
\end{corollary}
\begin{proof}
    We may assume that $X \subseteq \mathbb{P}^{N}$ is a closed subset for some $N \ge n$.
    When $N = n$, the only closed subset of $\mathbb{P}^{n}$ of dimension $n$ is itself hence we can take the identity map.
    Otherwise, when $N > n$, $X \subsetneq \mathbb{P}_{k}^{N}$.
    Pick $x\in \mathbb{P}_{k}^{N} \setminus X$, may assume $x = [1, 0, \cdots, 0]$.
    Take $\pi: \mathbb{P}_{k}^{N} \dashrightarrow \mathbb{P}_{k}^{N - 1}$ to be the projection from $x$.
    Then $\pi\big{|}_{X}: X \rightarrow \mathbb{P}_{k}^{N - 1}$ is a morphism.
    \par
    For all $p\in \mathbb{P}_{k}^{N - 1}$, $\pi\big{|}_{X}^{-1}(p) = X \cap L_{p}$,
    where $L_{p}$ is the line connecting $p$ and $x$.
    Since $L_{p} \not\subset X$, there exists $f\in I(X)$ which is not in $I(L_{p})$,
    Hence by Krull's Principal Ideal Theorem, $V(f) \cap L_{p}$ is of codimension $1$ in $L_{p}$ so that $L_{p} \cap X$ is a finite set.
    \par
    Let $Y = \overline{\pi(X)} \subseteq \mathbb{P}_{k}^{N - 1}$.
    By Theorem \ref{Theorem 4.5}, $\dim Y = \dim X = n$.
    Hence by induction, there exists dominant morphism $Y \rightarrow \mathbb{P}_{k}^{n}$ with all fibers finite.
    Now, $X \rightarrow Y \rightarrow \mathbb{P}_{k}^{n}$ is our desired morphism.
\end{proof}
\begin{corollary}
    Let $X, Y$ be varieties.
    Then $\dim X \times Y = \dim X + \dim Y$.
    \label{Corollary 4.6}
\end{corollary}
\begin{proof}
    May assume that $X, Y$ irreducible.
    Then there exist $f: X \rightarrow \mathbb{P}_{k}^{\dim X}$ dominant with all fibers finite and $g: X \rightarrow \mathbb{P}_{k}^{\dim Y}$ with all fibers finite.
    Consider $f \times g: X \times Y \overset{f \times g}{\rightarrow} \mathbb{P}_{k}^{\dim X + \dim Y}$,
    which is dominant with all fibers finite.
    Thus by Theorem \ref{Theorem 4.5}, $\dim X \times Y = \dim X + \dim Y$.
\end{proof}
\begin{corollary}
    Let $X,Y \subseteq \mathbb{P}_{k}^{n}$ be varieties.
    Then \\
    (1)If $X, Y$ are irreducible, then any nonempty irreducible component of $X \cap Y$ has dimension $\ge \dim X + \dim Y - n$. \\
    (2)If $X, Y$ are closed and $\dim X + \dim Y \ge n$, then $X \cap Y$ is nonempty.
    \label{Corollary 4.7}
\end{corollary}
\begin{proof}
    (1)Consider $p: \mathbb{A}_{k}^{n + 1} \rightarrow \mathbb{P}_{k}^{n}$ sending to $(x_{0}, \cdots, x_{n})$ to $[x_{0}, \cdots, x_{n}]$.
    Then for all $x\in \mathbb{P}_{k}^{n}$, denote $C^{0}(X) = p^{-1}(X)$ of dimension $\dim X + 1$ and $C^{0}(X \cap Y) = C^{0}(X) \cap C^{0}(Y)$.
    Note that $C^{0}(X) \cap C^{0}(Y) = (C^{0}(X) \times C^{0}(Y)) \cap \Delta_{\mathbb{A}_{k}^{n + 1}}$ and by Corollary \ref{Corollary 4.6}, $\dim (C^{0}(X) \times C^{0}(Y)) = \dim X + \dim Y + 2$,
    we get that $\dim C^{0}(X \cap Y) \ge \dim X + \dim Y - n + 1$ so that $\dim X \cap Y \ge \dim X + \dim Y - n$.
    \par
    (2)Take $\overline{C^{0}(X)} = C^{0}(X) \cup \{0\}$.
    Then 
    \begin{equation}
        \begin{split}
            \dim (\overline{C^{0}(X)} \cap \overline{C^{0}(Y)}) & \ge \dim (C^{0}(X) \cap C^{0}(Y)) \\
            & \ge n + 2 - n + 1 \\
            & = 1
        \end{split}
    \end{equation}
    Thus $\overline{C^{0}(X)} \cap \overline{C^{0}(Y)}$ is not just $\{0\}$ so that $X \cap Y \neq \varnothing$.
\end{proof}

\subsection{Morphism and dimension}
\label{subsection:Dimension_Theory_Morphism_and_dimension}

\begin{example}
    Recall blow up of $\mathbb{P}_{k}^{n}$ at $x$ with projection $p: Bl_{x}\mathbb{P}_{k}^{n} \rightarrow \mathbb{P}_{k}^{n}$.
    Then $p^{-1}(\mathbb{P}_{k}^{n} \setminus \{x\}) \overset{p}{\rightarrow} \mathbb{P}^{n} \setminus \{x\}$ is an isomorphism and $p^{-1}(x) \cong \mathbb{P}^{n - 1}$.
\end{example}
\begin{definition}
    Let $\varphi: X \rightarrow Y$ be a morphism.
    Define $X_{x} := \varphi^{-1}(\varphi(x))$ to be the fiber of $\varphi$ containing $x$.
\end{definition}
\begin{lemma}
    Let $Y$ be an affine variety of pure dimension $d$.
    Then for all $y\in Y$, there exist $F_{1}, \cdots, F_{d}\in A(Y)$ such that $y\in V(F_{1}, \cdots, F_{d})$ and $Y \cap V(F_{1}, \cdots, F_{d})$ is a finite set.
    \label{Lemma 4.2}
\end{lemma}
\begin{proof}
    Induct on $d$.
    For $d = 0$, $Y = V(I(Y))$ satisfies condition.
    For $d > 0$, note that for all $y\in Y$ and $y' \neq y$, there exists hyperplane $H_{y, y'}$ through $y$ but not $y'$.
    Hence the intersection of all hyperplanes through $y$ is just $\{y\}$.
    \par
    Similarly, since $d > 0$, for all irreducible component $Y_{i}$ of $Y$,
    we can find $y_{i}'\in Y_{i} \setminus \{y\}$ and there exists hyperplane $H_{y}$ such that $H_{y}$ does not contain $y_{i}$ for all $i$.
    Hence $Y_{i} \cap H$ is of dimension $d - 1$ for all $Y_{i}$ containing $y$.
    Then take $Y' = \cup_{Y_{i}\ni y} (Y_{i} \cap H)$ which is of pure dimension $d - 1$.
    By induction, done!
\end{proof}
\begin{theorem}
    Let $\varphi: X \rightarrow \mathbb{P}_{k}^{n}$ be a morphism.
    Consider the function $\delta: X \rightarrow \mathbb{N}$ sending $x$ to $\dim_{X_{x}} x$.
    Then $\delta$ is upper semicontinuous i.e. for all $r$, $X(r) := \{x\in X \big{|} \delta(x) \ge r\}$ is closed.
    Moreover, if $X$ is irreducible, then $\dim X = \dim \overline{\varphi(X)} + \min_{x\in X} \delta(x)$.
    \label{Theorem 4.6}
\end{theorem}
\begin{proof}
    Write $X = X_{1} \cup \cdots \cup X_{r}$, where $X_{i}$ are irreducible components.
    Denote $\varphi_{i}: X_{i} \hookrightarrow X \rightarrow \mathbb{P}_{k}^{n}$.
    Then $X_{x} = \cup_{i} (X_{x} \cap X_{i})$ and $X(r) = \cup_{i} X_{i}(r)$.
    Thus we may assume $X$ is irreducible.
    \par
    Set $Y = \overline{\varphi(X)}$, then $\varphi: X \rightarrow Y$ is dominant between irreducible varieties.
    For all $x\in X$ mapping to some $y\in Y$, by Lemma \ref{Lemma 4.2}, 
    there exists affine open neighbourhood $V$ of $y$ and regular functions $F_{1}, \cdots, F_{d}$ on $V$ vanishing at $y$ and $\dim (V \cap V(F_{1}, \cdots, F_{d})) = 0$,
    where $d = \dim Y$.
    Note that $\{y\}$ is a connected component of $V \cap V(F_{1}, \cdots, F_{d})$, we get $X_{x} = \varphi^{-1}(y)$ is a union of some irreducible components of $\varphi^{-1}(V) \cap V(\varphi^{\ast}F_{1}, \cdots, \varphi^{\ast}F_{d})$.
    \par
    By Theorem \ref{Theorem 4.2} and Krull's Principal Ideal Theorem, each irreducible component has dimension $\ge \dim X - d$.
    Thus for all $x\in X$, $\delta(x) \ge \dim X - d = \dim X - \dim Y$.
    Claim that there exists open dense subset $U \subseteq X$ such that for all $x\in U$, $\delta(x) = \dim X - \dim Y$.
    \par
    Firstly consider affine case.
    By similar argument as proof of Theorem \ref{Theorem 4.5}, we replace $X$ by graph $\Gamma_{f}$ of $f$ and there is a commutative diagram
    \begin{equation}
        \begin{tikzcd}
            \mathbb{A}_{k}^{n + m} \arrow[r] &
            \mathbb{A}_{k}^{n - 1 + m} \arrow[r] &
            \mathbb{A}_{k}^{n - 2 + m} \arrow[r] &
            \cdots \arrow[r] &
            \mathbb{A}^{m} \\
            X = Y_{0} \arrow[u, hookrightarrow] \arrow[r] &
            Y_{1} \arrow[u, hookrightarrow] \arrow[r] &
            Y_{2} \arrow[u, hookrightarrow] \arrow[r] &
            \cdots \arrow[r] &
            Y_{n} = Y \arrow[u, hookrightarrow]
        \end{tikzcd}
    \end{equation}
    For $Y_{i} \rightarrow Y_{i + 1}$, if $K(Y_{i})/K(Y_{i + 1})$ is transcendental,
    then fiber of all $y_{i + 1}\in Y_{i + 1}$ is isomorphic to projective line of dimension $1 = \dim Y_{i} - \dim Y_{i + 1}$.
    If $K(Y_{i})/K(Y_{i + 1})$ is algebraic, then a general fiber of $Y_{i} \rightarrow Y_{i + 1}$ is finite hence is of dimension $0 = \dim Y_{i} - \dim Y_{i + 1}$.
    Conclude that our claim holds for affine case.
    \par
    For general case, take affine open subset $U \subseteq Y$ and $V \subseteq \varphi^{-1}(U)$ affine open subset of $X$.
    Then there exists open dense subset $V' \subseteq V$ such that for all $x\in V'$, $\delta(x) = \dim V - \dim U = \dim X - \dim Y$.
    Since $V'$ is also dense in $X$, we complete proof of the claim.
    \par
    For $r \le \dim X - \dim Y$, we have seen that $X(r) = X$.
    For $r > \dim X - \dim Y$, there exists closed subvariety $F \subsetneq X$ such that $X(r) \subseteq F$.
    Consider $F(r)$, clearly $F(r) \subseteq X(r)$.
    Conversely, for all $x\in X(r)$, there exists some irreducible component $X'$ of $X_{x}$ containing $x$ such that $\dim X' \ge r$.
    Hence $X' \subseteq X(r) \subseteq F$ so that $X' \subseteq F_{x}$ and $x\in F(r)$.
    Now $F(r) = X(r)$.
    Note that $\dim F < \dim X$, by induction, we are done!
\end{proof}
\begin{corollary}
    Let $X$ be an irreducible variety, $\varphi: X \rightarrow Y$ dominant morphism.
    Then \\
    (1)for all $y\in \im \varphi$, any irreducible component of $\varphi^{-1}(y)$ has dimension $\ge \dim X - \dim Y$. \\
    (2)there exists nonempty open subset $U \subseteq Y$ such that for all $y\in U$, $\varphi^{-1}(y)$ is of pure dimension $\dim X - \dim Y$.
    \label{Corollary 4.8}
\end{corollary}
\begin{proof}
    (1)For all irreducible component $X'$ of $\varphi^{-1}(y)$,
    there exists $x'\in X'$ such that $X'$ is the only irreducible component containing $x'$.
    Hence $\dim X' = \dim_{\varphi^{-1}(y)} x' = \dim_{X_{x'}} x' \ge \dim X - \dim Y$.
    \par
    (2)We show that for all irreducible component $Z$ of $X(\dim X - \dim Y + 1)$, 
    the closure of its image in $Y$ is a proper subset i.e. $\overline{\varphi(Z)} \subsetneq Y$.
    For $x\in Z$ such that $x$ is not contained in any other irreducible component of $X(\dim X - \dim Y + 1)$,
    all irreducible components of $X_{x}$ have dimension $\ge \dim X - \dim Y + 1$.
    Thus $X' \subseteq X(\dim X - \dim Y + 1)$ and $X' \subseteq Z$ as $X'$ irreducible.
    If we take $U = X \setminus (\cup Z')$, where $Z'$ varies all irreducible components except $Z$,
    then for all $x\in U$, we get $\dim_{Z_{x}} x = \dim_{X_{x}} x \ge \dim X - \dim Y + 1$.
    \par
    Consider $\varphi\big{|}_{Z}: Z \rightarrow Y$.
    By claim in the proof of previous theorem, there exists open dense subset $U' \subseteq Z$ such that for all $x'\in U'$,
    $\delta(x') = \dim Z - \dim \overline{\varphi(Z)}$.
    As $Z$ is irreducible, $U \cap U' \neq \varnothing$ hence $\dim \overline{\varphi(Z)} \le \dim Z - \dim X + \dim Y - 1 < \dim Y$.
    Thus $\dim (\cup_{Z} \overline{\varphi(Z)}) = \max_{Z} \{\dim \overline{\varphi(Z)}\} < \dim Y$.
    And $W = (\cup_{Z} \overline{\varphi(Z)})^{c}$ is our desired set.
\end{proof}
\begin{remark}
    Part (2) in fact tells that for a dominant map from irreducible variety,
    there would be some nonempty open subset contained in the image.
\end{remark}
\begin{proposition}
    Let $\varphi: X \rightarrow Y$ be a closed morphism.
    Then $Y(r) := \{y\in Y \big{|} \dim \varphi^{-1}(y) \ge r\}$ is a closed subset of $Y$.
    \label{Proposition 4.4}
\end{proposition}
\begin{reason}
    Note that $Y(r) = \varphi(X(r))$.
\end{reason}
\begin{example}
    Consider $\varphi: \mathbb{A}_{k}^{3} \rightarrow \mathbb{A}_{k}^{3}$ sending $(x, y, z)$ to $(x, (xy - 1)y, (xy - 1)z)$.
    Easy to check that $\varphi$ is dominant and a general fiber of $\varphi$ is finite (the corresponding field extension is finite).
    Assume $\alpha = (\alpha_{1}, \alpha_{2}, \alpha_{3})\in Y(1)$.
    Then for all $(x, y, z)\in \varphi^{-1}(\alpha)$, we have that
    \begin{equation}
        \left\{
            \begin{aligned}
                & x = \alpha_{1} \\
                & (xy - 1)y = \alpha_{2} \\
                & (xy - 1)z = \alpha_{3}
            \end{aligned}
        \right.
        \label{equation:defining_equations_of_fiber_at_alpha}
    \end{equation}
    Since $\varphi^{-1}(\alpha)$ has dimension $\ge 1$, \ref{equation:defining_equations_of_fiber_at_alpha} cannot only have finitely many solutions.
    Hence $Y(1) \subseteq \{(a, 0, 0) \big{|} a \neq 0\}$.
    In fact, the equality holds so that $Y(1)$ is not closed in $\mathbb{A}_{k}^{3}$.
\end{example}
\begin{proposition}
    Let $\varphi: X \rightarrow Y$ be a closed dominant morphism.
    Assume that $Y$ is irreducible and all fibers of $\varphi$ are irreducible of dimension $r$.
    Then $X$ is irreducible and of dimension $\dim Y + r$.
    \label{Proposition 4.5}
\end{proposition}
\begin{proof}
    Assume $X_{i}$ are irreducible components of $X$.
    For all $y\in Y$, let $d_{i}(y)$ be the dimension of the fiber of $\varphi\big{|}_{X_{i}}$ at $y$.
    When $y\notin \im \varphi\big{|}_{X_{i}}$, set $d_{i}(y) = -1$.
    Then $r = \max_{i} \{d_{i}(y)\}$ so that $Y = \cup_{i} \{y\in Y \big{|} d_{i}(y) \ge r\}$.
    \par
    By previous Proposition, each $\{y\in Y \big{|} d_{i}(y) \ge r\}$ is closed.
    Since $Y$ is irreducible, there exists some $i$ such that $Y = \{y\in Y \big{|} d_{i}(y) \ge r\}$.
    Hence for all $y\in Y$, $\varphi^{-1}(y) = \varphi\big{|}_{X_{i}}^{-1}(y)$.
    Conclude that $X = X_{i}$ irreducible.
    For dimension, by the claim, we get $\dim X - \dim Y = r$.
\end{proof}

\subsection{Applications}
\label{subsection:Dimension_Theory_Applications}

\begin{lemma}[\textbf{\emph{Rigidity Lemma}}]
    Let $X$ be an irreducible projective variety over $Y$, $y_{0}\in Y$.
    Consider a morphism $\varphi: X \times Y \rightarrow Z$ between varieties.
    Assume that $\varphi(X \times \{y_{0}\})$ is one-point set $\{z_{0}\}$.
    Then for all $y\in Y$, $\varphi(X \times \{y\})$ is also one-point set.
    \label{Lemma 4.3} 
\end{lemma}
\begin{example}
    This example shows that projectiveness is necessary.
    For $\varphi: \mathbb{A}_{k}^{1} \times \mathbb{A}_{k}^{1} \rightarrow \mathbb{A}_{k}^{1}$ sending $(x, y)$ to $xy$,
    $\varphi$ contracts $\mathbb{A}_{k}^{1} \times \{0\}$ to $0$ but not contracts $\mathbb{A}_{k}^{1} \times \{y\}$ to $0$.
\end{example}
\begin{proof}
    Let $\Gamma_{f}$ be the graph of $\varphi$.
    Then $\Gamma_{\varphi} \subseteq (X \times Y) \times Z$ is a closed subvariety.
    Consider another projection $(X \times Y) \times Z \rightarrow Y \times Z$.
    Since $X$ is projective, $q$ is a closed morphism so that $q(\Gamma_{\varphi})$ is closed subset in $Y \times Z$.
    \par
    Consider $\psi: q(\Gamma_{\varphi}) \rightarrow Y \times Z \overset{p}{\rightarrow} Y$ which is clearly a surjection.
    Note that $\psi^{-1}(y_{0}) = (y_{0}, z_{0})$ of dimension $0 \ge \dim q(\Gamma_{\varphi}) - \dim Y$, we get that $\dim Y = \dim q(\Gamma_{\varphi})$.
    For all $\alpha\in q(\Gamma_{\varphi})$, we have that
    \begin{equation}
        \begin{split}
            \dim q\big{|}_{\Gamma_{\varphi}}^{-1}(\alpha) & \ge \dim \Gamma_{\varphi} - \dim q(\Gamma_{\varphi}) \\
            & = \dim X + \dim Y - \dim Y \\
            & = \dim X
        \end{split}
    \end{equation}
    while for all $\alpha\in q(\Gamma_{\varphi}) \subseteq Y \times Z$, $q^{-1}(\alpha) = X \times \{\alpha\}$ has dimension $X$.
    Since $q^{-1}(\alpha)$ is also of dimension $\dim X$,
    we get that $q^{-1}(\alpha) = q\big{|}_{\Gamma_{\varphi}}^{-1}(\alpha)$.
    In particular, for all $y$ mapping to $z$, take $\alpha = (y, z)$ and we get $q^{-1}(\alpha) = X \times \{y\} \times \{z\}$.
    Conclude that $X \times \{y\}$ is contracted to $\{z\}$.
\end{proof}
\begin{definition}
    Let $X$ be a variety.
    Assume that $X$ admits a group structure $(X, \cdot, \iota, e_{X})$,
    where multiplication $\cdot: X \times X \rightarrow X$, inverse $\iota: X \rightarrow X$ and identity map $e: \spec k \rightarrow X$ satisfy group structure.
    Then we call $X$ an algebraic group.
\end{definition}
\begin{example}
    (1)$(\mathbb{A}_{k}^{1}, +)$ \\
    (2)$({\mathbb{A}_{k}^{1}}^{\ast}, \cdot)$ \\
    (3)$(GL_{n}(k), \cdot)$ \\
    (4)$(O_{n}(k), \cdot)$, where $O_{n}(k) := \{A\in GL_{n}(k) \big{|} A^{\dagger}A = I_{n}\}$.
\end{example}
\begin{definition}
    Let $X$ be an irreducible projective algebraic group.
    We call that $X$ an abelian variety.
\end{definition}
\begin{theorem}
    The group operation of abelian variety is commutative.
    \label{Theorem 4.7}
\end{theorem}
\begin{proof}
    Let $X$ be an abelian variety.
    Consider map $\varphi: X \times X \rightarrow X$ sending $(x, y)$ to $x^{-1}yx$.,
    which is in fact a morphism as composition of morphisms
    \begin{equation}
        \begin{gathered}
            X \times X \\
            (x, y)
        \end{gathered}
        \rightarrow
        \begin{gathered}
            X \times X \times X \\
            (x, y, x)
        \end{gathered}
        \rightarrow
        \begin{gathered}
            X \times X \times X \\
            (x^{-1}, y, x)
        \end{gathered}
        \rightarrow
        \begin{gathered}
            X \\
            x^{-1}yx
        \end{gathered}
    \end{equation}
    Take the identity element $e_{X}$.
    Then $\varphi$ contracts $X \times \{e_{X}\}$.
    By rigidity lemma, for all $y$, $X \times \{y\}$ is contracted by $\varphi$.
    Hence $x^{-1}ax = e_{X}^{-1}ae_{X} = a$ and $ax = xa$.
\end{proof}
\begin{example}
    Let $X \subseteq \mathbb{P}_{k}^{2}$ be a general cubic curve.
    Then $X$ is a one-dimensional abelian variety whose group structure can be intuitively given as following.
    For $x_{1},x_{2}\in X$, the two points determines a line $L$ which should meet $X$ at three points counting multiplicity, denoted by $x_{1}, x_{2}, x_{3}$.
    We define the multiplication by sending $(x_{1}, x_{2})$ to $x_{3}$.
\end{example}
\begin{proposition}
    Let $X$ be an abelian variety, $Y$ an abelian group.
    Assume that $\varphi: X \rightarrow Y$ is a morphism such that $\varphi(e_{X}) = e_{Y}$.
    Then $\varphi$ is also a group homomorphism.
    \label{proposition 4.5}
\end{proposition}
\begin{proof}
    Consider map $\psi: X \times X \rightarrow Y$ sending $(x, x')$ to $\varphi(x)\varphi(x')\varphi(xx')^{-1}$,
    which is a morphism as composition of morphisms
    \begin{equation}
        \begin{gathered}
            X \times X \\
            (x, x')
        \end{gathered}
        \rightarrow
        \begin{gathered}
            X \times X \times X \\
            (x, x', xx')
        \end{gathered}
        \rightarrow
        \begin{gathered}
            Y \times Y \times Y \\
            (\varphi(x), \varphi(x'), \varphi(xx'))
        \end{gathered}
        \rightarrow
        \begin{gathered}
            Y \\
            \varphi(x)\varphi(x')\varphi(xx')^{-1}
        \end{gathered}
    \end{equation}
    As $\psi(X \times \{e_{X}\}) = \{e_{Y}\}$, by rigidity lemma,
    $\psi(X \times \{x'\})$ is a one-point set for all $x'\in X$.
    Hence we have
    \begin{equation}
        \varphi(x)\varphi(x')\varphi(xx')^{-1} = \varphi(e_{X})\varphi(x')\varphi(x')^{-1} = e_{Y}
    \end{equation}
    Thus $\varphi(x)\varphi(x') = \varphi(xx')$, done!
\end{proof}
\begin{definition}
    Let $X$ be a variety.
    A subset $S$ of $X$ is said to be constructible if it is a finite union of locally closed subsets i.e. $S = \cup_{i} (U_{i} \cap V_{i})$,
    where $U_{i}$ are open and $V_{i}$ are closed.
\end{definition}
\begin{remark}
    Moreover, $S$ can be written as $\cup_{i} S_{i}$, 
    where $S_{i}$ is open dense in $\overline{S_{i}}$ and $S_{i} = U_{i} \cap V_{i}$ for $V_{i}$ irreducible.
\end{remark}
\begin{theorem}[\textbf{\emph{Chevalley's Theorem}}]
    Let $\varphi:X \rightarrow Y$ be a morphism.
    Then the image of $X$ is a constructible subset of $Y$.
    \label{Theorem 4.8}
\end{theorem}
\begin{proof}
    We may assume $X$ is irreducible and $\varphi$ is dominant.
    By Corollary \ref{Corollary 4.8}, there exists nonempty open subset $U \subseteq Y$ such that for all $y\in Y$,
    fiber at $y$ is of dimension $\dim X - \dim Y \ge 0$ hence nonempty.
    Thus $U \subseteq \im(\varphi)$.
    \par
    Let $F = Y \setminus U$.
    Then $\varphi(X) = U \cup \varphi(\varphi^{-1}(F))$ where $\varphi^{-1}(F) \subsetneq X$ is a subvariety of lower dimension.
    We finish the proof by induction on $\dim X$.
    For the case that $\dim X = 0$, image is just a one-point set hence constructible.
\end{proof}

\subsection{Finite morphisms}
\label{subsection:Dimension_Theory_Finite_morphisms}

\begin{theorem}
    Let $\varphi: X \rightarrow Y$ be a morphism.
    Assume that all fibers of $\varphi$ is finite and $X$ is projective.
    Then for all $y\in Y$, there exists affine open neighbourhood $V$ of $y$ such that $U := \varphi^{-1}(V) $ is affine and $A(U)$ is finite over $A(V)$.
    \label{Theorem 4.9}
\end{theorem}
\begin{remark}
    More generally, in fact we have that quasi-finiteness $+$ universally closedness would imply finiteness.
    In addition, finiteness is local on the base i.e. if there exists an affine open covering satisfying the condition,
    then all affine open cover would satisfy the condition.
\end{remark}
\begin{example}
    Let $\varphi: \mathbb{A}_{k}^{1} \setminus \{0\} \hookrightarrow \mathbb{A}_{k}^{1}$ be the natural inclusion.
    Then $\varphi$ is quasi-finite but not finite since $k[t] \rightarrow k[t, \frac{1}{t}]$ is not finite homomorphism.
\end{example}
\begin{proof}
    Since $X$ is projective, replacing $X$ by graph, we get 
    $
    \begin{tikzcd}
        X \arrow[r, hookrightarrow] \arrow[dr, "\varphi"] &
        Y \times \mathbb{P}_{k}^{N} \arrow[d, "p"] \\
        & Y
    \end{tikzcd}
    $
    For all $y_{0}\in Y$, $\varphi^{-1}(y_{0})$ is a finite set.
    Hence we can choose a hyperplane $H$ such that $X \cap (\{y_{0}\} \times H) = \varnothing$.
    Note that $p$ is closed map, $p((Y \times H) \cap X)$ is a closed subset of $Y$ not containing $y_{0}$.
    There exists nonempty affine open neighbourhood $V\ni y_{0}$ such that $V \cap p((Y \times H) \cap X) = \varnothing$.
    \par
    Now we get $U := \varphi^{-1}(V) \subseteq Y \times H^{c}$, where $H^{c} \cong \mathbb{A}_{k}^{N}$ and there is a commutative diagram
    \begin{equation}
        \begin{tikzcd}
            U \arrow[r, hookrightarrow] \arrow[d] &
            Y \times \mathbb{A}_{k}^{N} \arrow[d, "p"] \\
            V \arrow[r, hookrightarrow] &
            Y 
        \end{tikzcd}
    \end{equation}
    Note that $U = X \cap (V \times H^{c})$ is a closed subvariety of $V \times \mathbb{A}_{k}^{N}$,
    where $V$ is affine, we get $V \times \mathbb{A}_{k}^{N}$ and $U$ both affine.
    Want to show $A(U)$ is finite $A(V)$-module.
    We already know that $A(U)$ is finitely generated as $A(V)$-algebra.
    Remains to show that $A(U)$ is integral over $A(V)$.
    \par
    Denote coordinate of $\mathbb{A}_{k}^{N}$ by $x_{1}, \cdots, x_{N}$.
    By Theorem \ref{Theorem 3.3}, $\varphi$ is universally closed.
    For all $a\in A(U)$, consider $U \times \mathbb{A}_{k}^{1} \overset{\varphi \times \identity}{\rightarrow} V \times \mathbb{A}_{k}^{1}$.
    Let $Z \subseteq U \times \mathbb{A}_{k}^{1}$ be the subvariety defined by $ax - 1$.
    Then $\varphi \times \identity(Z)$ is closed in $V \times \mathbb{A}_{k}^{1}$.
    \par
    Denote $J = I(\varphi \times \identity(Z))$.
    There is a ring homomorphism $A(V)[x]/J \rightarrow A(U)[x]/(ax - 1)$.
    For all $g\in J$, write $g = \sum_{i = 0}^{m} a_{i}x^{i}$, where $a_{i}\in A(V)$.
    Then $g$ is mapping to $(ax - 1)(\sum_{i = 0}^{m - 1} b_{i}x^{i})$ for some $b_{i}\in A(U)$.
    Hence $\varphi^{\ast} = ab_{i - 1} - b_{i}$, where $b_{m} = 0$.
    \par
    For $n > m$, let $f(T) = \sum_{i = 0}^{m} \varphi^{\ast}(a_{i})Y^{n - i}$.
    Clearly, we have that $f(a) = 0$.
    So it suffices to show that there exists $g\in J$ such that $a_{0} = 1$,
    which is equivalent to 
    \begin{equation}
        \begin{split}
            & (J, x) = A(V)[x] \\
            \iff & V(J) \cap V(X) = \varnothing \\
            \iff & \varphi \times \identity(Z) \cap (V \times \{0\}) = \varnothing \\
            \iff & Z \cap (U \times \{0\}) = \varnothing
        \end{split}
    \end{equation}
    By definition of $Z$, we are done!
\end{proof}

\section{Smooth Points and Smooth Morphisms}
\label{section:Smooth_Points_and_Smooth_Morphisms}

\subsection{Zariski tangent space}
\label{subsection:Smooth_Points_and_Smooth_Morphisms_Zariski_tangent_space}

\begin{definition}
    Let $X \subseteq \mathbb{A}_{k}^{N}$ be an affine variety, $p\in X$.
    Define the Zariski tangent space $T_{p}X$ of $X$ at $p$ to be the subspace defined by the linear equations $\{\sum_{i} \frac{\partial F}{\partial x_{i}}(p)x_{i} = 0 \big{|} \forall F\in I(X)\}$.
\end{definition}
\begin{definition}
    Let $X \subseteq \mathbb{A}_{k}^{N}$ be an affine variety, $p\in X$.
    A derivation of $X$ at $p$ is a $k$-linear map $D: A(X) \rightarrow k$ satisfying that for all $f,g\in A(X)$, $D(fg) = D(f)g(p) + f(p)D(g)$.
\end{definition}
For $X = \mathbb{A}_{k}^{n}$ with $A(X) = k[x_{1}, \cdots, x_{n}]$,
a derivation $D$ of $X$ at $p$ is determined by $D(x_{i})$ in the sense that $D = \sum_{i} D(x_{i})\frac{\partial f}{\partial x_{i}}(p)$.
Hence each $a\in T_{p}X$ correspondends to a derivation with $D(x_{i}) = a_{i}$.
\par
For general affine variety $X$ with $A(X) = k[x_{1}, \cdots, x_{n}]/I(X)$, 
a derivation $D$ of $X$ at $p$ is exactly a derivation of $\mathbb{A}_{k}^{n}$ at $p$ which kills all $F\in I(X)$,
whose corresponding point is contained in $T_{p}X$. 
\par
We can also identify $T_{p}X$ with $(\mathfrak{m}_{p}/\mathfrak{m}_{p}^{2})^{\vee}$,
where $\mathfrak{m}_{p}$ is the maximal ideal of $A(X)$ corresponding to $p$.
Given derivation $D: A(X) \rightarrow k$ at $p$, since $D(fg) = D(f)g(p) + f(p)D(g)$, 
we have that $D\big{|}_{\mathfrak{m}_{p}^{2}} = 0$ and $D\big{|}_{\mathfrak{m}_{p}}$ gives a $k$-linear function $\theta: \mathfrak{m}_{p}/\mathfrak{m}_{p}^{2} \rightarrow k$.
\par
Conversely, given $\theta: \mathfrak{m}_{p}/\mathfrak{m}_{p}^{2} \rightarrow k$.
Let $D_{\theta}: A(X) \rightarrow k$ be the map defined by $f \mapsto \theta(f - f(p))$.
Easy to $D_{\theta}$ satisfies Libniz's Law and hence is a derivation.
\begin{definition}
    Let $X$ be a variety, $p\in X$.
    Take an affine open neighbourhood $U$ of $p$.
    Define $T_{p}X := T_{p}U = (\mathfrak{m}_{p}/\mathfrak{m}_{p}^{2})^{\vee}$.
\end{definition}
\begin{remark}
    In fact, for $\mathcal{O}_{X, p} := A(X)_{\mathfrak{m}_{p}}$,
    if we denote $\widetilde{\mathfrak{m}_{p}} = \mathfrak{m}_{p}\mathcal{O}_{X, p}$,
    then we have a canonical isomorphism $\mathfrak{m}_{p}/\mathfrak{m}_{p}^{2} \cong \widetilde{\mathfrak{m}_{p}}/\widetilde{\mathfrak{m}_{p}}^{2}$.
    Hence the definition is independent to our choice of affine open neighbourhood.
\end{remark}
\begin{example}
    Let $X, Y$ be two varieties, $(x, y)\in X \times Y$.
    Then $T_{(x, y)}X \times Y = T_{x}X \oplus T_{y}Y$.
\end{example}

\subsection{Smooth points and singular points}
\label{subsection:Smooth_Points_and_Smooth_Morphisms_Smooth_points_and_singular_points}

\begin{proposition}
    Let $X$ be a variety.
    Then the function $X \rightarrow \mathbb{N}$ sending $x$ to $\dim_{k} T_{x}X$ is upper semicontinuous.
    \label{Proposition 5.1}
\end{proposition}
\begin{reason}
    By taking affine cover, we may assume $X$ is affine.
    Then $X \subseteq \mathbb{A}_{k}^{N}$ for some $N$.
    For all $x\in X$, $T_{x}X = \{(a_{1}, \cdots, a_{n}) \big{|} \sum_{i} a_{i}\frac{\partial F}{\partial x_{1}}, \forall F\in I(X)\}$.
    Assume $I(X) = (F_{1}, \cdots, F_{n})$.
    Then
    \begin{equation}
        \begin{split}
            & X_{r} = \{x\in X \big{|} \dim_{k} \ker (\frac{\partial F_{i}}{x_{j}}(x)) \ge r\} \\
            \iff & \rank(\frac{\partial F_{i}}{\partial x_{j}}(x)) \le N - r \\
            \iff & \text{ for all $(N - r + 1)$-minor, determinants vanish}
        \end{split}
    \end{equation}
    \vspace{-\baselineskip} % 减少一行间距
\end{reason}
\begin{proposition}
    Let $X$ be an irreducible variety.
    Then for all $x\in X$, $\dim_{k} T_{x}X \ge \dim X$ and there exists open dense subset $U$ of $X$ such that for all $x\in U$, $\dim_{k} T_{x}X = \dim X$.
    \label{Proposition 5.2}
\end{proposition}
\begin{proof}
    When $X \subseteq \mathbb{A}_{k}^{n + 1}$ is a hypersurface defined by an irreducible polynomial $0 \neq F$, then $\dim X = n$.
    For all $p\in X$, $T_{p}X$ is defined by a linear polynomial $\sum_{i} \frac{\partial F}{\partial x_{i}}(p)x_{1}$.
    Thus $\dim_{k} T_{p}X = n$ or $n + 1$.
    \par
    By previous proposition, $X_{n + 1}$ is a closed subset of $X$.
    If $X_{n + 1} = X$, then $\frac{\partial F}{\partial x_{i}}$ vanish at all $x\in X$ so that $\frac{\partial F}{\partial x_{i}} = 0$.
    Since $F$ is irreducible, we get $\characteristic k = p$ and $F = G(x_{1}^{p}, \cdots, x_{n + 1}^{p})$ for some polynomial $G$.
    As $k$ is algebraically closed, there exists $G_{1}$ such that $G_{1}^{p} = F$, contradicting to the assumption that $F$ is irreducible.
    Hence $X_{n + 1} \subsetneq X$ and $U = X_{n + 1}^{c}$ is our desired open subset.
    \par
    In general, we may assume that $X$ is affine and $n = \dim X = \trdeg (K(X)/k)$.
    By Proposition \ref{Proposition 2.4}, there exists some hypersurface $X' \subseteq \mathbb{A}_{k}^{n + 1}$ birational to $X$.
    Then there exists open dense subset $U \subseteq X$ and $U' \subseteq X'$ such that $U \overset{\sim}{\rightarrow} U'$.
    \par
    Suppose that $X_{m}^{c}$ is nonempty.
    Then since $X$ is irreducible, $X_{m}^{c} \cap U$ is also irreducible, contradicting to $X'_{n} = \varnothing$.
    Hence for all $x\in X$, $\dim_{k} T_{x}X \ge n$.
    And by discussion above, there exists open dense subset $U'' \subseteq X'$ such that for all $x'\in U''$, $\dim_{k} T_{x'}X' = n$.
    As $X'$ is irreducible, $U' \cap U''$ is nonempty whose inverse image in $X$ is our desired open dense subset.
\end{proof}
\begin{definition}
    Let $X$ be a variety.
    We say that $x\in X$ is a smooth point if $\dim_{k} T_{x}X = \dim_{X} x$.
    Otherwise, we would say that $x\in X$ is a singular point. 
\end{definition}
\begin{corollary}
    Let $X$ be a variety.
    If $X$ is smooth at $x$, then $\dim_{k} T_{x}X = \dim X$.
    Moreover, if $X$ is irreducible, then the converse also holds.
    \label{Corollary 5.1}
\end{corollary}
\begin{reason}
    By Proposition \ref{Proposition 5.2}, we immediately get that $\dim X \le \dim_{k} T_{x}X = \dim_{X} x \le \dim X$.
    When $X$ is irreducible, by Corollary \ref{Corollary 4.3}, we get $\dim_{X} x = \dim X$ for all $x$.
\end{reason}
\begin{remark}
    Combining Proposition \ref{Proposition 5.3} and Corollary \ref{Corollary 5.1},
    we get that for any irreducible variety $X$, there is always an open dense subset of $X$ consists of smooth points in $X$.
\end{remark}
\begin{example}
    (1)Let $X \subseteq \mathbb{A}_{k}^{3}$ be the subvariety defined by $(xy, yz)$.
    Then $\singular X = \{0\}$. \\
    (2)Let $X, Y$ be two varieties.
    Then $\singular (X \times Y) = (\singular X \times Y) \cup (X \times \singular Y)$. \\
    (3)Let $X \subseteq \mathbb{A}_{k}^{n + 1}$ be the subvariety defined by $F$.
    Then $\singular X$ is the subvariety defined by $(F, \frac{\partial F}{\partial x_{1}}, \cdots, \frac{\partial F}{\partial x_{n + 1}})$.
    In particular, if $X = \cup V(F_{i})$ where $F_{i}$ are irreducible,
    then for all $i \neq j$, $V(F_{i}) \cap V(F_{j}) \subseteq \singular X$. \\
    (4)Assume that $X \subseteq \mathbb{A}_{k}^{n}$ is an irreducible closed subvariety with $I(X) = (F_{1}, \cdots, F_{r})$.
    Then $p\in X$ is a smooth point if and only if $(\frac{\partial F_{i}}{\partial x_{j}}(p))$ has corank $\dim X$. \\
    (5)Assume that $X \subseteq \mathbb{P}_{k}^{n}$ is a hypersurface defined by homogeneous $F(x_{0}, \cdots, x_{n + 1})$.
    Then $\singular X$ is defined by $(\frac{\partial F}{\partial x_{0}}, \cdots, \frac{\partial F}{\partial x_{n}})$ since $\sum_{i} x_{i}\frac{\partial F}{\partial x_{i}} = dF$ where $d$ is the degree of $F$. 
\end{example}
\begin{lemma}
    Let $X$ be a variety.
    Then the set of singular points in $X$ is closed, denoted by $\singular X$.
    \label{Lemma 5.1}
\end{lemma}
\begin{reason}
    By taking affine cover, we may assume $X$ is affine.
    Hence by Jacobian criterion, we are done.
\end{reason}

\subsection{Regular rings}
\label{subsection:Smooth_Points_and_Smooth_Morphisms_Regular_rings}

\begin{lemma}
    Let $A$ be a noetherian local ring with residue field $k$, $\mathfrak{m}_{A}$ the maximal ideal.
    Then we always have that $\dim_{k} (\mathfrak{m}_{A}/\mathfrak{m}_{A}^{2}) \ge \dim A$.
    \label{Lemma 5.2}
\end{lemma}
\begin{reason}
    By Krull's Principal Ideal Theorem.
\end{reason}
\begin{definition}
    Let $A$ be a noetherian local ring with residue field $k$, $\mathfrak{m}_{A}$ the maximal ideal.
    We say that $A$ is a regular local ring if $\dim_{k} (\mathfrak{m}_{A}/\mathfrak{m}_{A}^{2}) = \dim A$.
\end{definition}
\begin{lemma}
    Let $A$ be a noetherian integral domain.
    Then $A$ is UFD if and only if all prime ideals of height $1$ are principal ideals.
    \label{Lemma 5.3}
\end{lemma}
\begin{theorem}
    A regular local ring is a UFD.
    \label{Theorem 5.1}
\end{theorem}
\begin{proof}
    Here is a sketch of the proof
    \begin{itemize}
        \item Let $R$ be an integral domain.
        Then $R$ is a UFD if and only if for all $f,g \neq 0$, $(f) : (g) := \{h\in R \big{|} gh\in (f)\}$ is a principal ideal.
        \item Let $R$ be a regular local of dimension $n$.
        Then $\widehat{R} \cong k[[x_{1}, \cdots, x_{n}]]$. 
        \item $R$ is UFD if and only if $\widehat{R}$ is UFD. 
    \end{itemize}
    \vspace{-1.5\baselineskip} % 减少一行间距
\end{proof}
\begin{corollary}
    Let $X$ be a variety.
    If $X$ is smooth at $x$, then $\mathcal{O}_{X, x}$ is a UFD.
    \label{Corollary 5.2}
\end{corollary}
\begin{example}
    Let $R = k[x_{1}, \cdots, x_{n}]/(x_{1}^{2} + \cdots + x_{n}^{2})$ and $\characteristic k \neq 2$.
    \begin{itemize}
        \item when $n = 2$, as $x_{1}^{2} + x_{2}^{2} = (x_{1} + ix_{2})(x_{1} - ix_{2})$, $R$ is not an integral domain.
        \item when $n = 3$, $x_{1}^{2} + x_{2}^{2} + x_{3}^{2}$ is irreducible in $R$, 
        but $R$ is not a UFD as $x_{3}^{2} = -(x_{1} + ix_{2})(x_{1} - ix_{2})$.
        \item when $n = 4$, $x_{1}^{2} + x_{2}^{2} + x_{3}^{2} + x_{4}^{2}$ is irreducible in $R$,
        but $R$ is not a UFD as $(x_{1} + ix_{2})(x_{1} - ix_{2}) = -(x_{3} + ix_{4})(x_{3} - ix_{4})$.
        \item when $n \ge 5$, $R$ is a UFD.
    \end{itemize} 
    \label{example:singularity_of_quotient_ring_by_square_sum}
\end{example}
\begin{lemma}[\textbf{\emph{Nagata's Lemma}}]
    Let $R$ be an integral domain.
    Assume $(x)$ is a prime ideal of $R$.
    Then if $R[\frac{1}{x}]$ is a UFD, so is $R$.
    \label{Lemma 5.4}
\end{lemma}
\begin{proof}
    Let $\mathfrak{p}$ be a prime ideal of $R$ of height $1$.
    If $\mathfrak{p} = (x)$, then $\mathfrak{p}$ is principal.
    Otherwise, $\mathfrak{p}R[\frac{1}{x}]$ is a prime ideal of $R[\frac{1}{x}]$ of height $1$.
    As $R[\frac{1}{x}]$ is a UFD, by Lemma \ref{Lemma 5.3} there exists some $y\in R[\frac{1}{x}]$ such that $\mathfrak{p}R[\frac{1}{x}] = (y)$.
    \par
    Assume that $y = y_{0}x^{n}$ where $y_{0}\in R \setminus (x)$.
    Note that $\mathfrak{p}R[\frac{1}{x}] \cap R = \mathfrak{p}$, 
    we get that $\mathfrak{p} = yR[\frac{1}{x}] \cap R = y_{0}R[\frac{1}{x}] \cap R = (y_{0})$ also principal.
    Again \ref{Lemma 5.3}, we conclude that $R$ is UFD.
\end{proof}
With Lemma \ref{Lemma 5.4}, we give a proof of the $n \ge 5$ case in Example \ref{example:singularity_of_quotient_ring_by_square_sum}.
\begin{lemma}
    Let $R = k[x_{1}, \cdots, x_{n}]/(x_{1}^{2} + \cdots + x_{n}^{2})$ and $\characteristic k \neq 2$.
    If $n \ge 5$, then $R$ is a UFD.
    \label{Lemma 5.5}
\end{lemma}
\begin{proof}
    Denote $x = x_{n - 1} + ix_{n}$ and $y = x_{n - 1} - ix_{n}$.
    As $x_{n - 1}^{2} + x_{n}^{2} = xy$, we can rewrite $R$ as $k[x_{1}, \cdots, x_{n - 2}, x, y]/(x_{1}^{2} + \cdots + x_{n - 2}^{2} + xy)$.
    Since $R/(x) \cong k[x_{1}, \cdots, x_{n - 2}, x, y]/(x_{1}^{2} + \cdots + x_{n - 2}^{2}, x) \cong k[x_{1}, \cdots, x_{n - 2}, y]/(x_{1}^{2} + \cdots + x_{n - 2}^{2})$ is integral domain, $(x)$ is a prime ideal of $R$.
    Taking localization at $x$, we get $R[\frac{1}{x}] \cong k[x_{1}, \cdots, x_{n - 2}, x]$ is a UFD.
    By Nagata's Lemma, $R$ is also a UFD.
\end{proof}
Theorem \ref{Theorem 5.1} is too hard to give a complete proof here. 
But we can prove for a much more easier but useful result.
\begin{lemma}
    A regular local ring is an integral domain.
    \label{Lemma 5.6}
\end{lemma}
\begin{proof}
    Induct on dimension of $R$.
    When $\dim R = 0$, we immediately get that $R$ is a field and hence an integral domain.
    Assume $\dim R > 0$ so that $\mathfrak{m}^{2} \subsetneq \mathfrak{m}$ otherwise by Nakayama's Lemma, $\mathfrak{m} = 0$.
    Since noetherian local ring has finitely many minimal prime ideals, 
    we can take some $x\in \mathfrak{m}^{2}/\mathfrak{m}$ and $x\notin \mathfrak{p}$ for all minimal prime ideal $\mathfrak{p}$.
    \par
    Then $R/(x)$ with maximal ideal $\mathfrak{m}/(x)$ would be a regular local ring of dimension $\dim R - 1$.
    By induction, $R/(x)$ is an integral domain so that $(x)$ is prime ideal of $R$.
    Hence $(x)$ should contain some minimal prime ideal $\mathfrak{q}$.
    For all $y\in \mathfrak{q}$, $y = ax$ for some $a\in R$.
    Note that $x\notin \mathfrak{q}$, we get $a\in \mathfrak{q}$ so that $\mathfrak{q} \subseteq (x)\mathfrak{q} \subseteq \mathfrak{m}\mathfrak{q}$.
    By Nakayama's Lemma, $\mathfrak{q}$ is zero ideal.
    Conclude that $R$ is an integral domain.
\end{proof}
\begin{proposition}
    Let $X$ be an irreducible variety of dimension $n$.
    Assume that $Y \subseteq X$ be a closed subvariety of pure dimension $n - 1$.
    If $y\in Y$ is a smooth point in $X$, 
    then there exists an open affine neighbourhood $U$ of $y$ in $X$ and irreducible regular function $f\in A(U)$ such that $I(U \cap Y) = (f)$ in $U$. 
    \label{Proposition 5.3}
\end{proposition}
\begin{proof}
    Assume $X$ is affine and $Y$ is irreducible.
    Then $Y$ is defined by a prime ideal $\mathfrak{p} \subseteq A(X)$ and $\mathfrak{p}\mathcal{O}_{X, y}$ is still a prime ideal of $\mathcal{O}_{X, y}$ of height $1$.
    By Corollary \ref{Corollary 5.2}, $\mathcal{O}_{X, y}$ is a UFD.
    Hence $\mathfrak{p}\mathcal{O}_{X, y}$ is generated by some irreducible element $\frac{f}{1}$.
    \par
    Consider $(\frac{f}{1})\mathcal{O}_{X, y} \cap A(X)$ which is an ideal of $A(X)$.
    Assume $f_{1}, \cdots, f_{r}$ generate the ideal.
    Then there exist $\frac{h_{i}}{g_{i}}\in \mathcal{O}_{X, y}$ such that $\frac{f_{i}}{1} = \frac{h_{i}}{g_{i}} \cdot \frac{f}{1}$.
    Take $g = \prod_{i} g_{i}$ and we have that $A(X)_{g} \cap (\frac{f}{1})\mathcal{O}_{X, y} = (\frac{f}{1})A(X)_{g}$.
    Let $U = D(g)$ and we are done.
\end{proof}
\begin{example}
    Let $X = V(x_{1}x_{2} - x_{3}x_{4}) \subseteq \mathbb{A}_{k}^{4}$.
    Easy to verify that $X$ is irreducible of dimension $3$.
    Consider $I = (x_{1}, x_{3})$.
    Then $V(I) \subseteq X$ is isomorphic to $\mathbb{A}_{k}^{2}$ of dimension $2$.
    But around origin $o$, $I\mathcal{O}_{X, o}$ is not principal.
    Hence smoothness is necessary in the previous proposition.
\end{example}

\subsection{Zariski's Main Theorem}
\label{subsection:Smooth_Points_and_Smooth_Morphisms_Zariski's_Main_Theorem}

\begin{theorem}
    Let $\varphi: X \rightarrow Y$ be a birational morphism between irreducible varieties of dimension $n$.
    Assume that $Y$ is smooth (or factorial).
    Then there exists open dense subset $U \subseteq Y$ such that 
    \begin{itemize}
        \item $\varphi^{-1}(U) \overset{\sim}{\rightarrow} U$
        \item If $X \setminus \varphi^{-1}(U)$ is not empty, write $X \setminus \varphi^{-1}(U) = \cup_{i} E_{i}$ as union of irreducible components.
        Then $\dim E_{i} = n - 1$ and $\dim \overline{\varphi(E_{i})} \le n - 2$.
    \end{itemize}
    \label{Theorem 5.3}
\end{theorem}
\begin{proof}
    For $y\in Y$, take an affine open neighbourhood $U_{y}$.
    For $x\in X$ mapping to $y$, take an affine open neighbourhood $U_{x} \subseteq \varphi^{-1}(U_{y})$.
    Then $\varphi$ induces map
    \begin{equation}
        \begin{tikzcd}
            A(U_{y}) \arrow[r, "\varphi^{\ast}"] \arrow[d, hookrightarrow] &
            A(U_{x}) \arrow[d, hookrightarrow] \\
            \mathcal{O}_{Y, y} \arrow[r, "\varphi^{\ast}"] \arrow[d, hookrightarrow] &
            \mathcal{O}_{X, x} \arrow[d, hookrightarrow] \\
            K(Y) \arrow[r, "\varphi^{\ast}", "\sim" swap] &
            K(X)
        \end{tikzcd}
    \end{equation}
    hence all the $\varphi^{\ast}$ are injections.
    \par
    For the case that $\varphi^{\ast}: \mathcal{O}_{Y, y} \rightarrow \mathcal{O}_{X, x}$ is surjective,
    assume that $t_{1}, \cdots, t_{m}$ generate $A(U_{x})$ over $k$.
    Then there exists $\frac{a_{i}}{b_{i}}\in \mathcal{O}_{Y, y}$ such that $\varphi^{\ast}(\frac{a_{i}}{b_{i}}) = t_{i}$,
    where $a_{i},b_{i}\in A(U_{y})$ and $b_{i}(y) \neq 0$.
    Take $b = \prod_{i} b_{i}$.
    Then $\varphi^{\ast}: A(U_{y})_{b} \rightarrow A(U_{x})_{\varphi^{\ast}(b)}$ is bijective and hence $D(\varphi^{\ast}(b))$ is isomorphic to $D(b)$.
    \par
    For the case that $\varphi^{\ast}$ is not surjective, for $t\in \widetilde{\mathfrak{m}_{x}}$ which is not in $\im \varphi^{\ast}$,
    there exists $a,b\in \widetilde{\mathfrak{m}_{y}}$ such that $\varphi^{\ast}(\frac{a}{b}) = t$.
    Since $Y$ is smooth, $\mathcal{O}_{Y, y}$ is UFD.
    We may assume that $a,b$ are coprime.
    After shrinking $U_{y}$, we may assume that $a,b\in A(U_{y})$.
    \par
    Take closed subvarieties $E_{1} = V(a)$ and $E_{2} = V(b)$ both of pure dimension $n - 1$.
    They have no common irreducible component through $y$.
    Shrinking $U_{y}$ again, we may assume $E_{1}$ and $E_{2}$ have no common irreducible component.
    Then $\dim (E_{1} \cap E_{2}) \le n - 2$.
    \par
    Consider $E = V(\varphi^{\ast}(b))$ of dimension $n - 1$.
    Note that $\varphi^{\ast}(a) = t\varphi^{\ast}(b)$, we get that $\varphi(E) \subseteq E_{1} \cap E_{2}$.
    Hence $\dim \overline{\varphi(E)} \le \dim (E_{1} \cap E_{2}) \le n - 2$, done!
\end{proof}
\begin{remark}
    Through this proof, it is clear that when apply Zariski's Main Theorem,
    we can take $U$ to be the unique maximal one.
\end{remark}
\begin{example}
    Let $X = V(x_{1}x_{2} - x_{3}x_{4}) \subseteq \mathbb{A}_{k}^{4}$.
    Then $\mathcal{O}_{X, o}$ is not UFD.
    Consider $\varphi: \mathbb{A}_{k}^{3} \rightarrow X$ sending $(y_{1}, y_{2}, y_{3})$ to $(y_{1}, y_{2}y_{3}, y_{2}, y_{1}y_{3})$.
    Note the field extension $\varphi^{\ast}: K(X) \rightarrow K(y_{1}, y_{2}, y_{3})$ is a isomorphism and hence $\varphi$ is birational morphism.
    Take $U \subseteq X$ defined by $\{(x_{1}, x_{2}, x_{3}, x_{4}) \big{|} (x_{1}, x_{3}) \neq 0\}$,
    then $\varphi^{-1}(U) = \{(y_{1}, y_{2}, y_{3}) \big{|} (y_{1}, y_{2}) \neq 0\}$ and $\dim (\mathbb{A}_{k}^{3} \setminus \varphi^{-1}(U)) = 1$.
    Thus Zariski's Main Theorem fails if without factoriality.
\end{example}
\begin{corollary}
    Let $\varphi: X \rightarrow Y$ be a quasi-finite birational morphism between irreducible varieties.
    Assume that $Y$ is smooth.
    Then $\varphi$ is an open immersion.
    In particular, $X$ is also smooth.
    \label{Corollary 5.3}
\end{corollary}
\begin{proof}
    By Zariski's Main Theorem, there exists an open dense subset $U \subseteq Y$ such that $\varphi^{-1}(U) \overset{\sim}{\rightarrow} U$.
    If $X \neq \varphi^{-1}(U)$, then for all $y\in \im \varphi \setminus U$, 
    since $\varphi$ is quasi-finite, $\varphi^{-1}(y) \subseteq X \setminus \varphi^{-1}(U)$ is union of some irreducible components of $X \setminus \varphi^{-1}(U)$.
    Hence all irreducible components $E_{i}$ of $X \setminus \varphi^{-1}(U)$ would be of dimension $0$.
    By Zariski's Main Theorem, $\dim \overline{\varphi(E_{i})} \le -1$ which is ridiculous, contradiction.
    Thus $X = \varphi^{-1}(U)$ and hence $\varphi$ is an open immersion.
\end{proof}
\begin{corollary}
    Let $X$ be a smooth irreducible variety and $\varphi: X \dashrightarrow \mathbb{P}_{k}^{n}$ be a rational map.
    Then $\varphi$ is defined over $U \subseteq X$ with $\codim_{X} (X \setminus U) \ge 2$.
    \label{Corollary 5.4}
\end{corollary}
\begin{proof}
    Let $U$ be the definition of domain of $\varphi$ and $\Gamma_{\varphi|_{U}}$ be the graph.
    Consider its closure $\overline{\Gamma_{\varphi|_{U}}}$ in $X \times \mathbb{P}_{k}^{n}$ and projection $p: \overline{\Gamma_{\varphi|_{U}}} \rightarrow X$.
    Then $p$ is a birational morphism.
    Note that continuous map preserves irreducibility, $\overline{\Gamma_{\varphi|_{U}}}$ is irreducible.
    \par
    By Zariski's Main Theorem, there exists open dense subset $V \subseteq X$ such that $p^{-1}(V) \overset{\sim}{\rightarrow} V$ and $\dim \overline{p(\overline{\Gamma_{\varphi|_{U}}} \setminus p^{-1}(V))} \le \dim X - 2$.
    Hence $\codim_{X} (X \setminus V) \ge \codim \overline{p(\overline{\Gamma_{\varphi|_{U}}} \setminus p^{-1}(V))} \ge 2$.
    While $p^{-1}$ is defined over $V$, we get $V \overset{p^{-1}}{\rightarrow} \overline{\Gamma_{\varphi|_{U}}} \rightarrow \mathbb{P}_{k}^{n}$ is a morphism coinciding with $\varphi\big{|}_{U}$ on $U \cap V$.
    Hence by definition, $V \subseteq U$ and $\codim (X \setminus U) \ge \codim (X \setminus V) \ge 2$.
\end{proof}
\begin{corollary}
    Let $C$ be a smooth curve.
    Assume that $\varphi: C \dashrightarrow \mathbb{P}_{k}^{n}$ is a rational map.
    Then $\varphi$ is a morphism.
    \label{Corollary 5.5}
\end{corollary}
\begin{reason}
    By Corollary \ref{Corollary 5.4}, $\varphi$ is defined over $U$ with $\codim (X \setminus U) \ge 2$.
    While $\dim C = 1$, we get $X \setminus U = \varnothing$ so that $\varphi$ is a morphism. 
\end{reason}

\subsection{Smooth morphisms}
\label{subsection:Smooth_Points_and_Smooth_Morphisms_Smooth_morphisms}

Let $\varphi: X \rightarrow Y$ be a morphism between varieties, $x\in X$, $y = \varphi(x)$.
We may assume that $X$ and $Y$ are affine.
Then there is an induced homomorphism $\varphi^{\ast}: A(Y) \rightarrow A(X)$ such that ${\varphi^{\ast}}^{-1}(\mathfrak{m}_{x}) = \mathfrak{m}_{y}$.
Recall that $T_{x}X = (\mathfrak{m}_{x}/\mathfrak{m}_{x}^{2})^{\vee}$ and $T_{y}Y = (\mathfrak{m}_{y}/\mathfrak{m}_{y}^{2})^{\vee}$.
There is also an induced tangent map $T_{x}\varphi: T_{x}X \rightarrow T_{y}Y$.
\begin{example}
    Consider $\varphi: \mathbb{A}_{k}^{m} \rightarrow \mathbb{A}_{k}^{n}$ sending $(x_{1}, \cdots, x_{m})$ to $(f_{1}, \cdots, f_{n})$,
    where polynomials $f_{1}, \cdots, f_{n}\in k[x_{1}, \cdots, x_{m}]$.
    For all $\alpha = (\alpha_{1}, \cdots, \alpha_{m})\in \mathbb{A}_{k}^{m}$,
    the tangent map $T_{\alpha}\varphi$ is defined by $(\frac{\partial f_{i}}{\partial x_{j}}(\alpha))$.
\end{example}
\begin{lemma}
    Given morphisms $X \overset{\varphi}{\rightarrow} Y \overset{\psi}{\rightarrow} Z$,
    then $T_{x}(\psi \circ \varphi) = T_{\varphi(x)}\psi \circ T_{x}\varphi$ for all $x\in X$.
    \label{Lemma 5.7}
\end{lemma}
\begin{lemma}
    Let $\varphi: X \rightarrow Y$ be a morphism, $x\in X$.
    Assume that $X'$ is the connected component of $X_{x}$ containing $x$.
    Then $T_{x}X' \subseteq \ker T_{x}\varphi$.
    \label{Lemma 5.8}
\end{lemma}
\begin{proof}
    As definition of Zariski tangent space is local, we may assume $X$ and $Y$ are affine.
    Now there is an induced homomorphism $\varphi^{\ast}: A(Y) \rightarrow A(X)$.
    Denote $y = \varphi(x)$.
    Since $A(X') = A(X)/I$ for some radical ideal containing $\varphi^{\ast}(\mathfrak{m}_{y})$,
    we get the composition $\mathfrak{m}_{y} \overset{\varphi^{\ast}}{\rightarrow} \mathfrak{m}_{x} \twoheadrightarrow \mathfrak{m}_{x}/I$ is zero map.
    Hence by definition, $T_{x}X' \subseteq \ker T_{x}\varphi$.
\end{proof}
\begin{remark}
    It is not necessary to take the connected component here.
    In fact, $T_{x}X_{x} \subseteq \ker T_{x}\varphi$.
\end{remark}
\begin{proposition}
    Let $\varphi: X \rightarrow Y$ be a morphism.
    Then the function $X \rightarrow \mathbb{N}$ sending $x$ to $\dim_{k} (\ker T_{x}\varphi)$ is upper semicontinuous i.e. for all $r\in \mathbb{N}$,
    $X(r) := \{x\in X \big{|} \dim_{k} (\ker T_{x}\varphi) \ge r\}$ is closed.
    \label{Proposition 5.4}
\end{proposition}
\begin{proof}
    We may assume that $X \subseteq \mathbb{A}_{k}^{n}$ is a subvariety defined by $F_{1}, \cdots, F_{s}$.
    And $\varphi$ is restriction of a polynomial map $\mathbb{A}_{k}^{n} \rightarrow \mathbb{A}_{k}^{m}$ sending $(x_{1}, \cdots, x_{n})$ to $(G_{1}, \cdots, G_{m})$.
    Note that for all $\alpha\in X$, $T_{\alpha}X$ is the variety defined by $\{\sum_{j} \frac{\partial F_{i}}{\partial x_{j}}(\alpha)x_{j}\}_{i}$ and $T_{\alpha}\varphi$ is defined by $(\frac{\partial G_{i}}{\partial x_{j}}(\alpha))$,
    we get that $\alpha\in X(r)$ if and only if rank of the following matrix is less than or equal to $n - r$
    \begin{equation}
        \begin{pmatrix}
            \frac{\partial F_{1}}{\partial x_{1}}(\alpha) & \cdots & \frac{\partial F_{1}}{\partial x_{n}}(\alpha) \\
            \vdots & \ddots & \vdots \\
            \frac{\partial F_{s}}{\partial x_{1}}(\alpha) & \cdots & \frac{\partial F_{s}}{\partial x_{n}}(\alpha) \\
            \frac{\partial G_{1}}{\partial x_{1}}(\alpha) & \cdots & \frac{\partial G_{1}}{\partial x_{n}}(\alpha) \\
            \vdots & \ddots & \vdots \\
            \frac{\partial G_{m}}{\partial x_{1}}(\alpha) & \cdots & \frac{\partial G_{m}}{\partial x_{n}}(\alpha)
        \end{pmatrix}
    \end{equation}
    which is a closed condition so that $X(r)$ is closed.
\end{proof}
\begin{example}
    Consider $\varphi: \mathbb{A}_{k}^{3} \rightarrow \mathbb{A}_{k}^{2}$ sending $(x, y, z)$ to $(z, x^{2}z + y)$.
    For $\alpha = (\alpha_{1}, \alpha_{2}, \alpha_{3})\in \mathbb{A}_{k}^{3}$, $T_{\alpha}\varphi: k^{3} \rightarrow k^{2}$ is defined by
    \begin{equation}
        \begin{pmatrix}
            0 & 0 & 1 \\
            2\alpha_{1}\alpha_{3} & 2\alpha_{2} & \alpha_{1}^{2}
        \end{pmatrix}
    \end{equation}
    which is of rank at least $1$.
    Hence $X(2) = \{\alpha\in \mathbb{A}_{k}^{3} \big{|} \dim_{k} (\ker T_{\alpha}\varphi) \ge 2\} = \{(0, 0, \alpha_{3}) \big{|} \alpha_{3}\in k\} \cup \{(\alpha_{1}, 0, 0) \big{|} \alpha_{1}\in k\}$.
    As $\varphi(X(2)) = \{(z, 0) \big{|} z\in k\}$, for all $(s, t)\in \mathbb{A}_{k}^{2}$ such that $t \neq 0$,
    by Jacobian criterion, we have that $\varphi^{-1}((s, t))$ is smooth.
\end{example}
\begin{definition}
    Let $X, Y$ be smooth irreducible varieties.
    We say a morphism $\varphi: X \rightarrow Y$ is smooth at $x\in X$ if $T_{x}\varphi$ is surjective.
    If $\varphi$ is smooth at all points in $X$, then we say $\varphi$ is smooth.
\end{definition}
\begin{remark}
    By Proposition \ref{Proposition 5.4}, for morphism $\varphi$ between smooth varieties, 
    the set of points where $\varphi$ is not smooth is a closed subset of $X$.
\end{remark}
\begin{lemma}
    Let $X$ be an irreducible variety, $Z$ irreducible subvariety of $X$ of codimension $r$.
    Assume $X$ and $Z$ are both smooth at $x\in Z$.
    Then there exists affine open neighbourhood $U\ni x$ and a regular sequence $f_{1}, \cdots, f_{r}$ in $A(U)$ such that $Z \cap U$ is an irreducible component of $V(f_{1}, \cdots, f_{r}) \subseteq U$.
    \label{Lemma 5.9}
\end{lemma}
\begin{proof}
    We may assume that $X$ is affine.
    Then $Z$ is defined by some prime ideal $\mathfrak{p}$ of $A(X)$ of height $r$.
    As $X$ and $Z$ are both smooth at $x$, we get that $A(X)_{\mathfrak{m}_{x}}$ and $(A(X)/\mathfrak{p})_{\mathfrak{m}/\mathfrak{p}}$ are both regular local ring.
    Since $A(X)$ is integral, by Theorem \ref{Theorem 4.2}, $\dim A(X)/\mathfrak{p} = \dim A(X) - r$.
    \par
    Assume $\dim A(X) = n$.
    Denote $\overline{\mathfrak{m}_{x}} = \mathfrak{m}_{x}/\mathfrak{p}$.
    Take a basis $a_{1}, \cdots, a_{n - r}$ of $\overline{\mathfrak{m}_{x}}/\overline{\mathfrak{m}_{x}}^{2} \cong \mathfrak{m}_{x}/(\mathfrak{m}_{x}^{2} + \mathfrak{m})$.
    Then we can extend it to a basis $a_{1}, \cdots, a_{n - r}, b_{1}, \cdots, b_{r}$ of $\mathfrak{m}_{x}/\mathfrak{m}_{x}^{2}$.
    Then for all $i$, there exist $\frac{\alpha_{ij}}{\beta_{ij}}\in A(X)_{\mathfrak{m}_{x}}$ and $c_{i}\in \mathfrak{m}_{x}^{2}$ such that $f_{i} := \frac{b_{i}}{1} + \sum_{j} \frac{\alpha_{ij}}{\beta_{ij}}\frac{a_{j}}{1} - \frac{c}{1}$ is contained in $\mathfrak{p}$.
    \par
    Now we get $f_{i}\in \mathfrak{p}$ which is clearly a regular sequence.
    Taking appropriate $s$, we get $f_{i}\in U := D(s)$.
    By Krull's Principal Ideal Theorem, minimal prime ideals in $V(f_{1}, \cdots, f_{r})$ would have height $r$.
    Note that $Z \cap U$ also has codimension $r$ and $Z \cap U \subseteq V(f_{1}, \cdots, f_{r})$.
    We conclude that $Z \cap U$ is an irreducible component of $V(f_{1}, \cdots, f_{r})$.
\end{proof}
\begin{proposition}
    Let $X$ and $Y$ be smooth irreducible varieties, $\varphi: X \rightarrow Y$ morphism.
    Then \\
    (1)All nonempty fiber of $\varphi$ has pure dimension $\dim X - \dim Y$ and $\varphi$ is dominant. \\
    (2)Assume that $Z \subseteq Y$ is a smooth closed subvariety,
    then $\varphi^{-1}(Z)$ is also smooth.
    \label{Proposition 5.5}
\end{proposition}
\begin{proof}
    (1)For all $x\in X$, take $X'$ to be the connected component of $X_{x}$ containing $x$.
    Then by Proposition \ref{Proposition 5.2} and Theorem \ref{Theorem 4.5}, $\dim_{k} T_{x}X' \ge \dim X' \ge \dim X - \dim \overline{\varphi(X)} \ge \dim X - \dim Y$.
    On the other hand, as $\varphi$ is smooth, $T_{x}\varphi$ is surjective.
    Hence $\dim_{k} T_{x}X' \le \dim_{k} (\ker T_{x}\varphi) = \dim_{k} T_{x}X - \dim_{k} T_{\varphi(x)}Y$.
    Since $X$ and $Y$ are both smooth, by Corollary \ref{Corollary 5.1}, we have that $\dim_{k} T_{x}X - \dim_{k} T_{\varphi(x)}Y = \dim X - \dim Y$.
    Thus all equalities hold and $\dim \overline{\varphi(X)} = \dim Y$ so that $\varphi$ is dominant.
    \par
    (2)We may assume that $Z \subseteq Y$ is smooth, irreducible and of codimension $r$.
    For all $z\in \varphi^{-1}(Z)$, $T_{z}\varphi(T_{z}\varphi^{-1}(Z)) \subseteq T_{\varphi(z)}Z$.
    Hence 
    \begin{equation}
        \begin{split}
            \dim_{k} T_{z}\varphi^{-1}(Z) & \le \dim_{k} T_{\varphi(z)}Z + \dim_{k} (\ker T_{z}\varphi) \\
            & = \dim Z + \dim X - \dim Y \\
            & = \dim X - r
        \end{split}
    \end{equation}
    On the other hand, for all $y\in Z$, by Lemma \ref{Lemma 5.9},
    there exist affine open neighbourhood $U\ni y$ and a regular sequence $f_{1}, \cdots, f_{r}$ in $A(U)$ such that $Z \cap U$ is an irreducible component of $V(f_{1}, \cdots, f_{r})$.
    As continuous map sending irreducible sets to irreducible sets, $\varphi^{-1}(Z \cap U)$ would be union of some irreducible components of $\varphi^{-1}(V(f_{1}, \cdots, f_{r})) = V(f_{1} \circ \varphi, \cdots, f_{r} \circ \varphi_{r})$.
    By Krull's Principal Ideal Theorem, those irreducible components should be of codimension at most $r$ in $\varphi^{-1}(U)$.
    \par
    Note that for each irreducible component $Z_{0}$ of $\varphi^{-1}(Z)$ and $z\in Z_{0}$ mapping to $y$,
    $Z_{0} \cap \varphi^{-1}(U)$ is an irreducible component of $\varphi^{-1}(Z \cap U)$.
    Hence $\codim Z_{0} = \codim Z_{0} \cap \varphi^{-1}(U) \le r$ in $X$.
    Since $X$ is irreducible, by Theorem \ref{Theorem 4.2}, $\dim_{k} T_{z}\varphi^{-1}(Z) \ge \dim \varphi^{-1}(Z) \ge \dim X - r$.
    Thus $\dim_{k} T_{z}\varphi^{-1}(Z) = \dim \varphi^{-1}(Z) = \dim X - r$.
    As $X$ is irreducible, $\varphi^{-1}(Z)$ is smooth.
\end{proof}
\begin{remark}
    For $\varphi: X \rightarrow Y$ between smooth projective varieties,
    it is very rare that $\varphi$ is smooth, except for some trivial case such as projection $Y \times Z \rightarrow Y$.
\end{remark}
\begin{proposition}[\textbf{\emph{Generic Smoothness}}]
    Assume $\characteristic k = 0$.
    Let $\varphi: X \rightarrow Y$ be a dominant morphism between irreducible varieties.
    Then there exist open dense subsets $V \subseteq Y$ and $U \subseteq X$ such that $\varphi\big{|}_{U}: U \rightarrow V$ is a smooth morphism.
    \label{Proposition 5.6}
\end{proposition}
\begin{proof}
    We can reduce the problem to affine case.
    Assume that $X \subseteq \mathbb{A}_{k}^{n}$ and $Y \subseteq \mathbb{A}_{k}^{m}$.
    Moreover, by replacing $X$ by graph of $\varphi$, $\varphi$ is restriction of the projection $\mathbb{A}_{k}^{n + m} \rightarrow \mathbb{A}_{k}^{m}$.
    By induction, we consider the case that $n = 1$ with projection $\mathbb{A}_{k}^{m + 1} \rightarrow \mathbb{A}_{k}^{m}$ sending $(x_{1}, \cdots, x_{m + 1})$ to $(x_{1}, \cdots, x_{m})$ and there are two cases.
    \par
    When $x_{m + 1}$ is transcendental over $K(Y)$, $X \cong Y \times \mathbb{A}_{k}^{1}$.
    Since $\varphi$ is the projection, $\varphi$ is smooth.
    When $x_{m + 1}$ is algebraic over $K(Y)$.
    Let $G(T) = a_{d}(y)T^{d} + a_{d - 1}(y)T^{d - 1} + \cdots + a_{0}(y)$ be the minimal polynomial of $x_{m + 1}$ over $K(Y)$, where $a_{i}\in K(Y)$.
    Taking $V = D(a_{d})$, we may assume that leading coefficient of $G(T)$ is $1$.
    \par
    By definition, $G(x_{m + 1})\in I(X)$.
    In addition, since $\varphi$ is the projection,
    by Euclidean division, we get $I(X) = (I(Y), G(x_{m + 1}))$.
    Take $V$ to be the set of smooth points in $Y$, by Lemma \ref{Lemma 5.1}, $V$ is 
\end{proof}
\begin{example}
    When $\characteristic k = p$, generic smoothness would fail.
    For instance, the tangent map of $\varphi: \mathbb{A}_{k}^{1} \rightarrow \mathbb{A}_{k}^{1}$ sending $x$ to $x^{p}$ is zero map everywhere.
\end{example}
\end{document}